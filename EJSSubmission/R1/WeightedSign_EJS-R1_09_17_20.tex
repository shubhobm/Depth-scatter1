\documentclass[ejs]{imsart}

\RequirePackage[OT1]{fontenc}
%\RequirePackage{amsthm,amsmath}
\RequirePackage[numbers]{natbib}
\RequirePackage[colorlinks,citecolor=blue,urlcolor=blue]{hyperref}
\usepackage{amssymb,amsmath,amsthm,color,outlines,subfigure,comment}
\usepackage[small]{caption}
%\usepackage{hyperref} % for linking references
%\hypersetup{colorlinks = true, citecolor = blue, urlcolor = blue}
\usepackage{graphics}
\usepackage{graphicx}
\usepackage{stackrel}
\usepackage{natbib}
\usepackage{soul}
%\usepackage[bordercolor=white,backgroundcolor=gray!30,linecolor=black,colorinlistoftodos]{todonotes}
%\newcommand{\rework}[1]{\todo[color=yellow,inline]{#1}}

% will be filled by editor:
\doi{10.1214/154957804100000000}
\pubyear{0000}
\volume{0}
\firstpage{0}
\lastpage{0}
%\arxiv{}

%%%settings
%\pubyear{2006}
%\volume{0}
%\issue{0}
%\firstpage{1}
%\lastpage{8}

%%%%%% put your definitions there:
\startlocaldefs
%\numberwithin{equation}{section}
%\theoremstyle{plain}
%\newtheorem{thm}{Theorem}[section]

\newcommand{\bfa}{{\bf{a}}}
\newcommand{\bfb}{{\bf{b}}}
\newcommand{\bfc}{{\bf{c}}}
\newcommand{\bfd}{{\bf{d}}}
\newcommand{\bfe}{{\bf{e}}}
\newcommand{\bff}{{\bf{f}}}
\newcommand{\bfg}{{\bf{g}}}
\newcommand{\bfl}{{\bf{l}}}
\newcommand{\bfp}{{\bf{p}}}
\newcommand{\bfq}{{\bf{q}}}
\newcommand{\bfr}{{\bf{r}}}
\newcommand{\bfs}{{\bf{s}}}
\newcommand{\bft}{{\bf{t}}}
\newcommand{\bfu}{{\bf{u}}}
\newcommand{\bfv}{{\bf{v}}}
\newcommand{\bfw}{{\bf{w}}}
\newcommand{\bfx}{{\bf{x}}}
\newcommand{\bfy}{{\bf{y}}}
\newcommand{\bfz}{{\bf{z}}}

\newcommand{\bfA}{{\mathbf{A}}}
\newcommand{\bfB}{{\mathbf{B}}}
\newcommand{\bfC}{{\mathbf{C}}}
\newcommand{\bfD}{{\mathbf{D}}}
\newcommand{\bfE}{{\mathbf{E}}}
\newcommand{\bfF}{{\mathbf{F}}}
\newcommand{\bfG}{{\mathbf{G}}}
\newcommand{\bfH}{{\mathbf{H}}}
\newcommand{\bfI}{{\mathbf{I}}}
\newcommand{\bfL}{{\mathbf{L}}}
\newcommand{\bfP}{{\mathbf{P}}}
\newcommand{\bfR}{{\mathbf{R}}}
\newcommand{\bfS}{{\mathbf{S}}}
\newcommand{\bfT}{{\mathbf{T}}}
\newcommand{\bfU}{{\mathbf{U}}}
\newcommand{\bfV}{{\mathbf{V}}}
\newcommand{\bfW}{{\mathbf{W}}}
\newcommand{\bfX}{{\mathbf{X}}}
\newcommand{\bfY}{{\mathbf{Y}}}
\newcommand{\bfZ}{{\mathbf{Z}}}

\newcommand{\BA}{{\mathbb{A}}}
\newcommand{\BB}{{\mathbb{B}}}
\newcommand{\BC}{{\mathbb{C}}}
\newcommand{\BE}{{\mathbb{E}}}
\newcommand{\BD}{{\mathbb{D}}}
\newcommand{\BF}{{\mathbb{F}}}
\newcommand{\BG}{{\mathbb{G}}}
\newcommand{\BH}{{\mathbb{H}}}
\newcommand{\BI}{{\mathbb{I}}}
\newcommand{\BJ}{{\mathbb{J}}}
\newcommand{\BK}{{\mathbb{K}}}
\newcommand{\BL}{{\mathbb{L}}}
\newcommand{\BP}{{\mathbb{P}}}
\newcommand{\BR}{{\mathbb{R}}}
\newcommand{\BS}{{\mathbb{S}}}
\newcommand{\BT}{{\mathbb{T}}}
\newcommand{\BU}{{\mathbb{U}}}
\newcommand{\BV}{{\mathbb{V}}}
\newcommand{\BW}{{\mathbb{W}}}
\newcommand{\BX}{{\mathbb{X}}}
\newcommand{\BY}{{\mathbb{Y}}}
\newcommand{\BZ}{{\mathbb{Z}}}


\newcommand{\BCOV}{{\mathbb{COV}}}

\newcommand{\colrit}{\color{red} \it}
\newcommand{\colrbf}{\color{red} \bf}
\newcommand{\colbit}{\color{blue} \it}
\newcommand{\colbbf}{\color{blue} \bf}
\newcommand{\colmit}{\colm \it}
\newcommand{\colmbf}{\colm \bf}


\newcommand{\bfalpha}{{\boldsymbol{\alpha}}}
\newcommand{\bfbeta}{{\boldsymbol{\beta}}}
\newcommand{\bfepsilon}{{\boldsymbol{\epsilon}}}
\newcommand{\bftheta}{{\boldsymbol{\theta}}}
\newcommand{\bfgamma}{{\boldsymbol{\gamma}}}
\newcommand{\bfmu}{{\boldsymbol{\mu}}}
\newcommand{\bfpi}{{\boldsymbol{\pi}}}


\newcommand{\bfBeta}{{\boldsymbol{\Beta}}}
\newcommand{\bfGamma}{{\boldsymbol{\Gamma}}}
\newcommand{\bfOmega}{{\boldsymbol{\Omega}}}
\newcommand{\bfSigma}{{\boldsymbol{\Sigma}}}
\newcommand{\bfPhi}{{\boldsymbol{\Phi}}}


\newcommand{\bfZero}{\bf 0}
\newcommand{\bfOne}{\bf 1}
\newcommand{\iid}{\stackrel{i.i.d.}{\sim}}
\newcommand{\raro}{\rightarrow}

\newcommand{\cA}{\mathcal{A}}
\newcommand{\cB}{\mathcal{B}}
\newcommand{\cC}{\mathcal{C}}
\newcommand{\cD}{\mathcal{D}}
\newcommand{\cE}{\mathcal{E}}
\newcommand{\cF}{\mathcal{F}}
\newcommand{\cH}{\mathcal{H}}
\newcommand{\cI}{\mathcal{I}}
\newcommand{\cM}{\mathcal{M}}
\newcommand{\cN}{\mathcal{N}}
\newcommand{\cS}{\mathcal{S}}
\newcommand{\cX}{\mathcal{X}}
\newcommand{\cY}{\mathcal{Y}}

\def\baq#1\eaq{\begin{align}#1\end{align}}
\def\ban#1\ean{\begin{align*}#1\end{align*}}
\newcommand{\draro}{ \stackrel{{\mathcal D}}{\Rightarrow} }

\def\bredbf#1\eredbf{{\color{red}{\bf ???? #1 ????}}}
\def\BeginRedComment#1\EndRedComment{({\color{red}{\bf Comment:} {\it #1}}) }
\def\BeginBlueComment#1\EndBlueComment{({\color{blue}{\bf Comment:} {\it #1}}) }

\DeclareMathOperator*{\ve}{vec}
\DeclareMathOperator*{\diag}{diag }
\DeclareMathOperator*{\supp}{supp }
\DeclareMathOperator*{\Tr}{Tr}
\DeclareMathOperator*{\argmin}{arg\,min}
\DeclareMathOperator*{\argmax}{arg\,max}
\DeclareMathOperator*{\Th}{^{\text{th}}}

\makeatletter
\newcommand{\opnorm}{\@ifstar\@opnorms\@opnorm}
\newcommand{\@opnorms}[1]{%
  \left|\mkern-1.5mu\left|\mkern-1.5mu\left|
   #1
  \right|\mkern-1.5mu\right|\mkern-1.5mu\right|
}
\newcommand{\@opnorm}[2][]{%
  \mathopen{#1|\mkern-1.5mu#1|\mkern-1.5mu#1|}
  #2
  \mathclose{#1|\mkern-1.5mu#1|\mkern-1.5mu#1|}
}
\makeatother

\newtheorem{Theorem}{Theorem}[section]
\newtheorem{Lemma}[Theorem]{Lemma}
\newtheorem{Corollary}[Theorem]{Corollary}
\newtheorem{Proposition}[Theorem]{Proposition}
\newtheorem{Conjecture}[Theorem]{Conjecture}
\theoremstyle{definition} \newtheorem{Definition}[Theorem]{Definition}

\endlocaldefs



\begin{document}

\begin{frontmatter}
\title{On Weighted Multivariate Sign Functions}
\runtitle{Weighted Sign Functions}
%\thankstext{T1}{Footnote to the title with the `thankstext' command.}

\begin{aug}
\author{\fnms{Subhabrata} \snm{Majumdar}\thanksref{t1}\ead[label=e1]{subho@research.att.com}}
\and
\author{\fnms{Snigdhansu} \snm{Chatterjee}\thanksref{t3}\ead[label=e2]{chatt019@umn.edu}}

\address{School of Statistics,\\
University of Minnesota,\\
313 Ford Hall,\\
224 Church Street S.E., \\
Minneapolis 55455, USA\\
%usually few lines long\\
\printead{e1,e2}}

%\author{\fnms{Third} \snm{Author}
%\ead[label=e3]{third@somewhere.com}
%\ead[label=u1,url]{www.foo.com}}
%
%\address{Address of the Third author\\
%usually few lines long\\
%usually few lines long\\
%\printead{e3}\\
%\printead{u1}}

\thankstext{t1}{Currently at AT\&T Labs Research}
%\thankstext{t2}{First supporter of the project}
\thankstext{t3}{Corresponding author}
\runauthor{Majumdar and Chatterjee}

\affiliation{University of Minnesota and University of Minnesota}

\end{aug}

\begin{abstract}
Multivariate sign functions are often used for robust estimation and inference. We propose using data dependent weights in association with such functions. The proposed weighted sign functions  retain desirable robustness properties, while significantly improving efficiency in estimation and 
inference compared to unweighted multivariate sign-based methods. Using weighted signs, we demonstrate methods of robust location estimation and robust principal component analysis. We extend the scope of using robust multivariate methods to include robust sufficient dimension reduction and functional outlier detection. Several numerical studies and real data applications demonstrate the efficacy of the proposed methodology.
\end{abstract}

\begin{keyword}[class=MSC]
\kwd[Primary ]{62G35}
%\kwd{62G35}
\kwd[; secondary ]{62H25, 62H20, 62G99}
\end{keyword}

\begin{keyword}
\kwd{Multivariate sign}
\kwd{Principal component analysis}
\kwd{Data depth}
\kwd{Sufficient dimension reduction}
%\kwd{Multivariate sign}
%\kwd{\LaTeXe}
\end{keyword}
\tableofcontents
\end{frontmatter}

\input{WeightedSign_EJS_Introduction_09_17_20_1}
\section{The Weighted Spatial Median}
\label{Sec:WSQuantiles}

Suppose the open unit sphere in $\cX$ is given by 
$\textrm{int} {\cX}_{0; 1} = \{ x \in \cX: | x | < 1 \}$, and let
 $u \in \textrm{int} {\cX}_{0; 1}$. We also fix the set of probability measures $\cM$, 
 and select $\BF \in \cM$. Consider a random element $X \in \cX$, and define the 
 function $\Phi (q; X, u, \BF) = W (X, \BF) \bigl\{ | X - q | 
+ \langle u, X - q \rangle \bigr\}$. 
We define the $(u, \BF)$-th \textit{weighted spatial quantile} of $\cX$ 
as the minimizer $q (u, \BF) \in \cX$ of the expectation of $\Phi (q; X, u, \BF)$, 
that is
\baq
\Psi (q; u, \BF) = \BE \Bigl[ W (X, \BF) \bigl\{ | X - q | 
+ \langle u, X - q \rangle \bigr\} \Bigr] = \BE \Phi (q; X, u, \BF).
\label{eq:WeightedQuantile}
\eaq
%
%\rework{
\noindent
This is a natural generalization of the spatial median \cite{ref:JASA96862_Chaudhuri, ref:Biometrika48414_Haldane,ref:AoS97435_Koltchinskii} ($W(X, \BF) \equiv 1$ and $u = {\bf 0}_p$), or more general spatial quantiles \cite{ref:AoS17591_Cardotetal_Median_HilbertSpace, ref:AoS141203_ChakrabortyChaudhuri_Banach_Quantile, ref:Bernoulli152308_Minsker_Median_Banach} ($W(X, \BF) \equiv 1$). We assume that 
$\Phi (q; X, u, \BF)$ is convex in $q$ for $\BF$-almost all values of $x \in \cX$. 
 

In what follows, for brevity we elaborate only the case of the \textit{weighted spatial median} (thus $\Psi (q; 0, \BF) = \BE \bigl[ W (X, \BF) | X - q | \bigr] $) for the case of finite dimensional $\cX$. Specifically, we demonstrate the utility of using the {\it weighted} spatial median as opposed to using the traditional, unweighted versions found in literature. The analysis and computation of the general weighted spatial quantiles will largely follow by extending the results of the above cited references, and we postpone details to a future study.
%}

Assume that we have independent and identically distributed observations $X_{1}, \ldots, X_{n} \in \cX$, and the sample weighted spatial median is computed by minimizing $\Psi_{n} (q; 0, \BF) = \sum_{i = 1}^{n} W (X_{i}, \BF) | X_{i} - q | $, and is denoted by $\hat{q}_{nW}$, the second subscript is in acknowledgement that the weight function is used. We denote the unweighted version of this estimator, ie, the case where $W (X, \BF) \equiv 1$ as $\hat{q}_{n}$. Assume the following technical conditions:

\begin{description}
\item[(A1)] $\Psi (q; 0, \BF)$ is finite for all $q \in \cX \subseteq \BR^{p}$ 
and has a unique minimizer $q_{0}$. 

\item[(A2)] $\Psi (q; 0, \BF)$ is twice differentiable at $q_{0}$ and the second derivative is positive definite. 
 
\item[(A3)] $ {\frac{\partial^{2}} {\partial q^{2}}} \Psi (q; 0, \BF)$
exists for all $q$ in a neighborhood of $q_{0}$, and we use the notations
\ban
\Psi_{1W}   & = 
\Bigl( {\frac{\partial} {\partial q}} \Psi (q_{0}; 0, \BF) \Bigr)
 \Bigl( {\frac{\partial} {\partial q}} \Psi (q_{0}; 0, \BF) \Bigr)^{T}, \\
\Psi_{2W}   & = {\frac{\partial^{2}} {\partial q^{2}}} \Psi (q_{0}; 0, \BF). 
\ean
 \end{description}
 These assumptions are very broad-based and general. The first one essentially requires 
 the existence of a population parameter, the second one requires that the minimization 
 approach is meaningful in the population, and the third one essentially requires that the 
 weight function has a finite variance. No further restrictions are placed on the tuning parameter $\BF$ or the choice of the weight function. 
 
 \begin{Theorem}\label{Thm:WSMedian}
 Under assumptions [A1]-[A3], we have 
 \ban 
 n^{1/2} (\hat{q}_{nW} - q_{0}) & \draro N(0, \Psi_{2W}^{-1} \Psi_{1W} \Psi_{2W}^{-1}).
 \ean
 \end{Theorem}

%\rework{
Thus, under very standard regularity conditions, the sample weighted spatial median 
is consistent and is asymptotically normal. Theorem~\ref{Thm:WSMedian}  can be proved in several different ways. Here we use techniques following \cite{ref:AoS891631_Haberman, ref:AoS921514_Niemiro}. Specifically, following Theorem 4 in \cite{ref:AoS921514_Niemiro}, which traces back to \cite{ref:AoS891631_Haberman} with slightly relaxed conditions, we get
%
\begin{align*}
n^{1/2} (\hat{q}_{nW} - q_{0}) =
- \frac{\Psi_{2W}}{\sqrt n} 
\sum_{i=1}^n \nabla \Phi (q; X_{i}, u, \BF)
+ o_P(1),
\end{align*}
%
where $\nabla \Phi (q; X_{i}, u, \BF)$ is any measurable subgradient of 
$\Phi (q; X_{i}, u, \BF)$. 
Theorem~\ref{Thm:WSMedian} follows by applying the central limit theorem.

\paragraph{Remark.} Note that the technical conditions for the result presented in Theorem~\ref{Thm:WSMedian} is one of several alternatives that can conceived, and the scope of this result is broader than what is presented above. First, note that if $\BF$ 
and $\BF_{X}$ are different and the weights are not a function of $q$, 
a situation that may arise in hypothesis testing problems where the weights are based 
on the null distribution, the convexity of $\Phi (q; X_{i}, u, \BF)$ follows automatically 
and is not an assumption. Second, even if $\Phi (q; X_{i}, u, \BF)$ is not convex but sufficiently smooth, we can have a central limit theorem, for example, by using techniques similar to \cite{ref:CBose_AoS05414}. 
Choices of $\BF$ other than $\BF_{X}$, e.g. \cite{StatPaper18},  may lead to interesting interpretations of $W(. ,\BF)$ and the resulting location estimator and will be explored further in future.


\subsection{Asymptotic efficiency of weighted spatial median}
Let $V_{W} = \Psi_{2W}^{-1} \Psi_{1W} \Psi_{2W}^{-1}$ be the asymptotic 
variance of $\hat{q}_{n,W}$ from Theorem~\ref{Thm:WSMedian}, where we use the 
subscript ``${}_{W}$'' to denote that this depends on the weight function. 
We use the notation 
$V_{1}$ for the case where $W (x, \BF) \equiv 1$, that is, all weights are one.
The asymptotic relative efficiency of two statistics is the $p$-th root of 
the reciprocals of their determinants. That is, 
\ban 
ARE (\hat{q}_{nW}, \hat{q}_{n}) = \Bigl\{\frac{det(V_{1})}{det(V_{W})} \Bigr\}^{1/p}.
\ean
One easy result from Theorem~\ref{Thm:WSMedian} is that under reasonable conditions the
asymptotic relative efficiency of the weighted spatial median over the unweighted version
is always greater than one. We document this in the following corollary:

\begin{Corollary}
\label{Cor:ARE}
Assume that the weight function $W (X, \BF)$ is bounded above by $k$ for some $k > 0$, and the matrices $\Psi_1 = \BE S(X; q_0) S^T(X; q_0)$ and $\Psi_{1W}$ are positive definite. Then
%
\ban
ARE (\hat{q}_{nW}, \hat{q}_{n}) \geq
\frac{ \lambda_{\min} (\Psi_1) \lambda_{\min}^2 (\Psi_{2W})}
{k \lambda_{\max} (\Psi_{1W}) \lambda_{\max}^2 (\Psi_2)}
\ean
%
Consequently, if $k / \lambda_{\min}^2 (\Psi_{2W}) < \lambda_{\min} (\Psi_1) /(\lambda_{\max} (\Psi_{1W}) \lambda_{\max}^2 (\Psi_2))$ then this asymptotic relative efficiency is larger than 1. 
\end{Corollary}

\begin{proof}[Proof of Corollary~\ref{Cor:ARE}]
Using the facts that $det (AB) = det(A) det(B)$ for square matrices $A,B$ and $det (A^{-1}) = 1/det(A)$ for non-singular $A$, we write
%
\ban
\frac{det (V_1)}{det (V_W)} &= det( \Psi_2^{-1} \Psi_1 \Psi_2^{-1} )
det( \Psi_{2 W} \Psi_{1 W}^{-1} \Psi_{2 W} )\\
&= det (\Psi_1) det (\Psi_{1 W}^{-1}) [ det(\Psi_2^{-1}) det(\Psi_{2 W}) ]^2
\ean
%
The result follows, using the facts that $\det(A) \geq \lambda_{\min} (A)$ and $\det(A^{-1}) \geq 1/\lambda_{\max} (A)$, and the upper bound on $W$.
\end{proof}

%A context where 
It is possible to make the above asymptotic relative efficiency to tend to infinity with 
$p$, for example by choosing $k = \exp\{- p^{2} \}$. Such tending to infinity efficiencies 
are also achievable for fixed $k$,  for example, in the context of some hypothesis tests.
%ing problems.
%f $|X - q_{0}|^{2}$ is a Gamma random 
%variable with shape parameter $ = 2 + \exp\{- p^{2} \}$. 
We leave it to future work to 
study such efficiencies more generally. 

We may wish to further explore the conditions of Corollary~\ref{Cor:ARE}. 
Let us concentrate on the case of where the distribution of $X$ is spherically symmetric. Following \cite{ref:Fangetal90_Book}:

\begin{Definition}
A $p$-dimensional random vector $X$ is said to elliptically distributed if there exist a vector $\mu \in \BR^p$, a positive semi-definite matrix $\Sigma \in \BR^{p \times p}$ and a function 
$\phi: (0, \infty) \rightarrow \BR$ such that the characteristic function of $X$ is $\exp \{ i t^{T} \mu \} \phi (\bft^T \Sigma \bft)$ for $\bft \in \BR^p$.
\end{Definition}

There are several alternative formulations, see the above reference and citations within it for details.

\begin{Corollary}\label{Cor:Elliptic_ARE}
Assume that $\BF \equiv \BF_X$ is an elliptically symmetric distribution, with location parameter $\mu = q_{0}$ and the covariance matrix $\Sigma$ satisfies the following conditions:
%
\begin{enumerate}
%\item $Tr(\Sigma^4) = o( Tr^2 (\Sigma^2) )$,

%\item $\frac{Tr^4(\Sigma)}{ Tr^2 (\Sigma^2)} \exp \left\{ - \frac{Tr^2 (\Sigma)}
%{ 128 \lambda_{\max}^2 (\Sigma)} \right\} = o(1)$,

\item $\exp \left\{ - \frac{Tr^2 (\Sigma)}{ 256 \lambda_{\max}^2 (\Sigma)} \right\} = 
o\left( \min\left( \frac{\lambda_{\max}(\Sigma) }{Tr( \Sigma)},
\frac{\lambda_{\min}(\Sigma)}{\lambda_{\max}(\Sigma)} \right) \right)$,

\item $\lambda_{\max}(\Sigma) = o( Tr(\Sigma))$.
\end{enumerate}

Also assume that the weight function is (a) bounded above by some $W_{\max} > 0$, (b) affine invariant, i.e. $ W (Ax + b, A\BF + b) = W (x, \BF)$ for $A \in \BR^{p \times p}, b \in \BR^p$, and (c) satisfies the following:
%
\begin{equation}\label{eqn:CorElliptic_ARE_eqn}
\frac{W_{\max}}{[ \BE | X - q_0|^{-1} W(X, \BF)]^2 } <
\frac{\lambda_{\min} (\Psi_1)}{\lambda_{\max} (\Psi_1) [ \BE | X - q_0|^{-1}]^2]}.
\end{equation}
%
Then we have $ARE( \hat q_{n,W}, \hat q_n) > 1$.
\end{Corollary}

The conditions 1 and 2 in Corollary~\ref{Cor:Elliptic_ARE} are due to \cite{ref:JASA151658_WangPengLi}, and ensure that the eigenvalues of $\Sigma$ are bounded away from 0 and $\infty$. Corollary~\ref{Cor:Elliptic_ARE} can be proved using Corollary~\ref{Cor:ARE}, the fact that for elliptical distributions $\Psi_1$ is non-singular \cite{ref:SPL12765_Taskinenetal}, then using and expanding upon Lemma A.5 in \cite{ref:JASA151658_WangPengLi}. We give the technical details in the supplementary material.

Obtaining affine invariant weight functions is not challenging. Most functions arising in the context of data depths are affine invariant. Corollary~\ref{Cor:Elliptic_ARE} implies that there is a wide frame of distributions and  choices of weight functions where there is a benefit to considering weighted spatial medians. In fact, Corollary~\ref{Cor:Elliptic_ARE} can be established under just 
location invariance condition on weights. We use the stronger affine invariance condition
here only because the subsequent sections use weights based on affine invariant 
data depth functions. Also note that in case \eqref{eqn:CorElliptic_ARE_eqn} does not hold for the original form of a weight function, one can scale all weights by an appropriate constant to reduce the value in the left hand side of \eqref{eqn:CorElliptic_ARE_eqn} to satisfy the condition.

In actual computations, in place of $W (x, \BF)$ we propose using $W (x, \BF_{n})$ where
$ \BF_{n}(z) = n^{-1} \sum_{i = 1}^{n} \prod_{j=1}^{p}
\cI (X_{ij} \leq z_j ), z \in \BR^p
$ is the empirical distribution function. Up to first order asymptotics, the analysis remain unchanged from the above for this modified weight function.

\subsection{Examples of affine invariant weights}

We now illustrate some specific choices of weight functions that are compatible with the 
conditions of Corollary~\ref{Cor:Elliptic_ARE} and the results in the rest of this paper.
 These arise as easy transformations 
of \textit{data-depth functions}. A data depth function 
is defined on $\cX \times \cF$, where $\cF$ is  a fixed set of probability measures.
The main property of a data-depth function is that for every probability measure 
$\BF \in \cF$, there exists a constant 
$\mu_{\BF} \in  \cX$ such that for any $t \in [ 0, 1]$ and $x \in \cX$,
\baq 
D ( \mu_{\BF} ; \BF ) \geq D ( \mu_{\BF} + t ( \bfx - \mu_{\BF} ); \BF ). 
\label{eq:Peripherality}
\eaq
That is, for every fixed $\BF$, the data-depth function achieves a supremum at 
$\mu_{F}$, and is non-decreasing in every direction away from $\mu_{F}$, thus providing 
a  center-outward partial ordering of points in $\cX$. There are generally several 
algebraic and analytic properties assumed for data-depth functions to elicit interesting 
mathematical properties, see for example \cite{ref:DIMACS061_Serfling, ref:AoS00461_ZuoSerfling} for details. 


The spherically symmetric case of an elliptical distribution is realized with $\Sigma = \sigma^{2} \BI_{p}$ for some $\sigma^{2} > 0$. We fix the notation $Z = \Sigma^{-1/2} (X - \mu)$, and let 
$Z \sim \BF_{Z}$. Note that $\BF_{Z}$ is a spherically symmetric distribution and hence
depends only on $|z|$, and $\BE Z = \mathbf{0}_{p} \in \BR^{p}$ and $\BV Z = \BI_{p}$. Taking affine invariant data depth functions as weights ensures that $W(X, \BF) = W(Z, \BF_Z)$. It is now easy to show that results in this paper are valid for the weight functions (with appropriate scaling to satisfy \eqref{eqn:CorElliptic_ARE_eqn})
($i$) $W_{HSD} (X) = \BF_{Z} (| Z |)$ derived from the \textit{half-space depth},  
($ii$) $W_{MhD} (X) = |Z|^{2}/(1 + | Z |^2)$ derived from the \textit{Mahalanobis depth}, 
and ($iii$) $W_{PD} (X) = |Z|/(1+ | Z |/MAD(\BF_{Z}))$, where $MAD$ stands for median 
absolute deviation, derived from the \textit{projection depth}. We omit the technical 
details. These three weight functions give a center-inward partial ordering, thus essentially 
quantifying \textit{peripherality} instead of \textit{depth}.
Note however, that our results below are of much more general form, and 
these three special choices of weights only serve as important illustrative examples 
to achieve desirable robustness and efficiency balance in data analysis.



\section{A Robust Measure of  Dispersion} 
\label{Sec:WSDispersion1}

From this section onwards, we assume that $\cX = \BR^{p}$, that is, the support of the 
random variable under study is the $p$-dimensional Euclidean plane, and the data 
$X_{1}, \ldots, X_{n}$ are independent and identically distributed from an 
elliptical distribution $\BF$ with parameters $\mu$ and $\Sigma$.  We also 
assume that $X_{1}$ is absolutely continuous, with $\BP [ | X_{1}| = 0] = 0$, 
and that $\Sigma$ is positive definite. This  eliminates technicalities arising from
 rank deficient cases, and makes the weight functions  affine invariant.
Thus, we essentially restrict the rest of this paper to the same  framework as in 
Corollary~\ref{Cor:Elliptic_ARE}. Most of the results below generalize to the case where 
$\cX$ is a separable Hilbert space, however additional technicalities are involved, as in 
\cite{ref:AoS112852_Balietal_PCA_Functional_Robust}, and will be considered 
in a future project. 


Consider the spectral decomposition of $\Sigma$ given by
 $\Sigma = \Gamma\Lambda\Gamma^T$, where $\Gamma$ is an orthogonal matrix 
 and $\Lambda$ is diagonal with positive diagonal elements $\lambda_1 \geq \ldots \geq 
 \lambda_p$.
Also denote the $i$-th eigenvector of $\Sigma$ by 
$\bfgamma_{i} = (\gamma_{i, 1}, \ldots, \gamma_{i, p})^T$ for $1 \leq i \leq p$. Thus, the 
$i$-th column of $\Gamma$ is $\bfgamma_{i}$.
  In the rest of this paper we use the notation 
 $\Sigma^{-1/2} = \Lambda^{-1/2} \Gamma^{T}$, and 
 hence $Z = \Lambda^{-1/2} \Gamma^T (X - \mu)$. 
Recall from Section~\ref{Sec:WSQuantiles} that we use the notation $\BF_{Z}$ for 
the distribution of $Z$, and that $\BF_{Z}$ is a spherically symmetric distribution 
and hence depends only on $|z|$.
Additionally, to simplify notations, for any random variable $X \sim \BF$, we occasionally 
use the abbreviated notation $W (X) \equiv W (X, \BF)$. Note that $W (X)$ is a 
\textit{random weight}, and takes the same value as $W (Z, \BF_{Z})$.  
Also, as noted in  Corollary~\ref{Cor:Elliptic_ARE}, 
 $W (Z)$ is a function of $|Z|$ only. 
 We additionally assume that $\BE W^{2} (X) < \infty$.
 
It is convenient to write 
\ban
X = \mu + R \Gamma \Lambda^{1/2}  U
\ean
where $U$  is a random variable uniformly distributed on the unit sphere 
${\cS}_{0; 1} = \{ x \in \cX: | x | = 1 \}$ and $R$ is another random variable independent 
of $U$ satisfying $\BE R^{2} = p$. Note that  $Z = R U$, and $|Z| = R$, $Z/|Z| = U$. 
Then we have
\ban 
S (X; \mu) & = |X - \mu|^{-1} (X - \mu) 
%\\
%& 
= | \Lambda^{1/2} R U|^{-1} R \Gamma  \Lambda^{1/2}  U\\
& =  | \Lambda^{1/2} U|^{-1} \Gamma \Lambda^{1/2}  U 
%\\
%& 
=  | \Lambda^{1/2} Z|^{-1} \Gamma \Lambda^{1/2}  Z
\ean


The rest of this paper is built on the following transformed random variable:
\baq
\tilde{X} = W (X, \BF) S(X; \mu) \equiv W (Z, \BF_{Z})  
| \Lambda^{1/2} Z|^{-1} \Gamma \Lambda^{1/2}  Z. 
\label{eq:tildeX}
\eaq
We define the notation $\BS (X; \mu) =  S (X; \mu) S( X; \mu)^T$.
Then, we define the following dispersion  parameter:
\baq
\tilde{\Sigma} = \BE \tilde{X} \tilde{X}^{T}
=  \BE W^{2} (X, \BF) \BS (X; \mu). 
\label{eq:tildeSigma}
\eaq
 In the following Theorem, we establish that the eigenvectors of ${\Sigma}$ and 
 $\tilde{\Sigma}$ are identical, although their eigenvalues may be different.
  
\begin{Theorem}
\label{Thm:WSVariance}
Under the conditions listed above, we have
$\tilde{\Sigma} = \Gamma \tilde{\Lambda} \Gamma^{T}$, where 
$\tilde{\Lambda} = \Lambda^{1/2} \BE W^{2} (X) 
	\BE UU^{T}/(U^{T}\Lambda U) \Lambda^{1/2}$
is a diagonal matrix.  Thus, the eigenvectors of ${\Sigma}$ and 
 $\tilde{\Sigma}$ are identical.
\end{Theorem}

\begin{proof}[Proof of Theorem~\ref{Thm:WSVariance}]

Fix any index $i \in \{1, \ldots, p \}$. Consider the vector $\tilde{U}$ such that 
\ban 
\tilde{U}_{j} = \left\{ \begin{array}{ll}
U_{j} & \text{ if } j \ne i, \\
- U_{i} & \text{ if } j = i. 
\end{array}
\right. 
\ean
Then $\tilde{U}$ and $U$ have the same distribution, and note that 
$U^{T} \Lambda U = \tilde{U}^{T} \Lambda \tilde{U}$ almost surely. Consequently, 
for any $j \ne i$ we have
\ban
\BE {\frac{U_{i} U_{j}}{U^{T} \Lambda U}}  
& = \BE {\frac{\tilde{U}_{i} \tilde{U}_{j}}{\tilde{U}^{T} \Lambda \tilde{U}}} 
%\\
%& 
= - \BE {\frac{{U}_{i} {U}_{j}}{U^{T} \Lambda U}}.
\ean
Consequently, 
$\BE S (X; \mu) S(X; \mu)^{T} = \Gamma \Lambda_{S} \Gamma^{T}$, 
as established in Theorem~1 of \citet{ref:SPL12765_Taskinenetal}.

Also, since the weight $W(X)$ is a function of $|Z| = R$, we have that $W(X)$ is 
independent of $S (X; \mu)$. Consequently, we have  
  \ban
  \tilde{\Sigma} & = \BE \tilde{X} \tilde{X}^{T} 
%  \\
%&
 = \BE W^{2} (X) S (X; \mu) S(X; \mu)^{T} \\
& =  \BE W^{2} (X)  \BE S (X; \mu) S(X; \mu)^{T} 
%\\
%& 
= \Gamma \Lambda_{W} \Gamma^{T}, 
\ean
where $\Lambda_{W}$ is a diagonal matrix.

\end{proof}


We now discuss the properties of the sample version $\widehat{\tilde{\Sigma}}$ 
computed from $\bfX$. 
In practice, we cannot obtain $W (x) \equiv W (x, \BF)$, 
and consequently use  $W (x, \BF_{n})$ instead. We assume the following conditions:
\begin{enumerate}
\item \textbf{Bounded weights:} The weights $W (\cdot, \cdot)$ are bounded functions. 

\item \textbf{Uniform convergence:} 
\ban 
\sup_{x \in \cX} | W (x, \BF_{n}) - W (x, \BF) | \rightarrow 0 
\ean 
almost surely as $n \rightarrow \infty $.

\item \textbf{Smoothness under perturbation:}
For all $\BF \in \cF$, there exists a $\delta > 0$, possibly depending on $\BF$, such that 
for any $\epsilon \in (0, \delta)$ 
\ban 
\sup_{x \in \cX} \Bigl| W \bigl( x, \BF \bigr) - 
W \bigl( x, (1 - \epsilon)\BF + \epsilon \delta_{x} \bigr) \Bigr| \leq 
\epsilon.
\ean 
 \end{enumerate}
 In the above, $\delta_{x}$ denotes point mass at $x$. 
These property are easily satisfied under for  weight functions  derived 
from standard depth functions, for example, $W_{HSD} (\cdot)$, 
$W_{MhD} (\cdot)$ and $W_{PD} (\cdot)$ discussed earlier.



The following result allows us to use the empirical, plug-in weights and an 
estimated location parameter in the 
weighted dispersion estimator. A natural choice for the location parameter estimator 
is the solution to $\sum_{i = 1}^{n} \tilde{X}_{i} = 0$. 

\begin{Lemma} \label{Lemma:lemma1}
Assume that $\BE \| X - \mu \|^{-4} < \infty$. Also assume that 
we have a location estimator  $\hat{\mu}_n$ satisfying 
$\BE \|\hat{\mu}_n  - \mu \|^{4} = O (n^{-2}) $. Then
\ban
\frac{1}{n} \sum_{i=1}^{n} W_{n}^{2} (X_{i}, \BF_{n}) \BS (X_{i}; \hat{\mu}_{n})  
= \frac{1}{n} \sum_{i=1}^{n} W^{2} (X_{i}, \BF) \BS (X_{i}; {\mu})
+ R_n,
\ean
%
where for any $c \in \BR^{p}$ such that $| c | = $, we have 
$\BE  c^{T} R_{n} c = o (n^{-1})$.
\end{Lemma}

\begin{proof}[Proof of Lemma~\ref{Lemma:lemma1}]
This proof is mostly algebra, and we provide a sketch of the main arguments.
We have
\ban 
& \frac{1}{n} \sum_{i=1}^{n} W_{n}^{2} (X_{i}, \BF_{n}) \BS (X_{i}; \hat{\mu}_{n}) \\
= & \frac{1}{n} \sum_{i=1}^{n} W^{2} (X_{i}, \BF) \BS (X_{i}; {\mu})
+ \frac{1}{n} \sum_{i=1}^{n}  
\bigl\{ W_{n}^{2} (X_{i}, \BF_{n})  -   W^{2} (X_{i}, \BF) \bigr\} \BS (X_{i}; {\mu}) \\
& + \frac{1}{n} \sum_{i=1}^{n} W^{2} (X_{i}, \BF) \bigl\{ 
\BS (X_{i}; \hat{\mu}_{n}) - \BS (X_{i}; {\mu}) \bigr\} \\
& + \frac{1}{n} \sum_{i=1}^{n}  
\bigl\{ W_{n}^{2} (X_{i}, \BF_{n})  -   W^{2} (X_{i}, \BF) \bigr\} 
\bigl\{ \BS (X_{i}; \hat{\mu}_{n}) - \BS (X_{i}; {\mu}) \bigr\}\\
& = \frac{1}{n} \sum_{i=1}^{n} W^{2} (X_{i}, \BF) \BS (X_{i}; {\mu})
+ T_{2} + T_{3} + T_{4}. 
\ean

Using the stated technical conditions, we can now show that 
$\BE  c^{T} T_{j} c = o (n^{-1})$ for $j = 2, 3, 4$. For illutration, we present
the case for $T_{2}$ below. 

Notice that the $(j, k)$-th element of $T_{2}$ is given by 
\ban 
n^{-1} \sum_{i=1}^{n}  | X_{i} - \mu |^{-2}
\bigl\{ W_{n}^{2} (X_{i}, \BF_{n})  -   W^{2} (X_{i}, \BF) \bigr\} 
(X_{i, j} - \mu_{j}) (X_{i, k} - \mu_{k}), 
\ean
and hence 
\ban 
& c^{T} T_{2} c  = \sum_{j, k} c_{j} c_{k}T_{2, j, k} \\
& = n^{-1} \sum_{i=1}^{n}  | X_{i} - \mu |^{-2}
\bigl\{ W_{n}^{2} (X_{i}, \BF_{n})  -   W^{2} (X_{i}, \BF) \bigr\} 
\bigl( \sum_{j} c_{j} (X_{i, j} - \mu_{j}) \bigr)^{2} \\
& \leq 
M n^{-1} \sum_{i=1}^{n}  | X_{i} - \mu |^{-2}
\bigl\{ | W_{n} (X_{i}, \BF_{n})  -   W (X_{i}, \BF) | \bigr\} 
\bigl( c^{T} (X_{i} - \mu) \bigr)^{2} \\
& \leq 
M n^{-1} \sum_{i=1}^{n}  | X_{i} - \mu |^{-2}
\bigl\{ | W_{n} (X_{i}, \BF_{n})  -   W_{n} (X_{i}, \BF_{n, -i})  | \bigr\} 
\bigl( c^{T} (X_{i} - \mu) \bigr)^{2} 
\\ & \hspace{1cm}
+ M n^{-1} \sum_{i=1}^{n}  | X_{i} - \mu |^{-2}
\bigl\{ | W_{n} (X_{i}, \BF_{n, -i})  -   W (X_{i}, \BF) | \bigr\} 
\bigl( c^{T} (X_{i} - \mu) \bigr)^{2} 
\\
& =  M n^{-1} \sum_{i=1}^{n}  T_{2 1 i}  + M n^{-1} \sum_{i=1}^{n}   T_{2 2 i} \\
& = T_{21} + T_{22}.
\ean
Let $H (X_{i}) = | X_{i} - \mu |^{-2} \bigl( c^{T} (X_{i} - \mu) \bigr)^{2}$, and notice 
that $H (X_{i}) \leq 1$ almost surely for $|c| = 1$.  
Now notice that conditional on $X_{i}$ except for a null set $A_{i}$ (possibly depending 
on $X_{i}$) we have  
$T_{2 1 i} \leq n^{-1}  H (X_{i}) $. Thus, except for a null set 
$A_{1} \cap \ldots \cap A_{n}$, $T_{21} \leq M n^{-2}  H (X_{i})$ and the conclusion 
follows for this part.

The argument for $T_{22}$ follows a similar argument.
\end{proof}




Let $vec~(\BS (X; \mu))$ be the vectorized version of  $\BS (X; \mu)$.
We are now in a position to state the result for consistency of the sample 
version of $\tilde{\Sigma}$,

\begin{Theorem} \label{Theorem:CLT1}
Assume the conditions of Lemma~\ref{Lemma:lemma1}. Then
\ban
& n^{1/2} \sum_{i = 1}^{n} \Bigl( 
W_{n}^{2} (X_{i}, \BF_{n}) ~vec~(\BS (X_{i}; \hat{\mu}_{n})) 
- \BE W^{2} (X_{i}) ~vec~(\BS (X_{i}; \mu)) \Bigr)\\
& \hspace{0.5cm} \draro
N_{p^2} \bigl( 0, V_{W} \bigr),
\ean
where $V_{W} = \BV [W^{2} (X, \BF) ~vec~(\BS (X; {\mu})) ] $.
\end{Theorem}

The asymptotic normality  follows from our assumptions and as a direct consequence 
of Lemma~\ref{Lemma:lemma1}. Incidentally, an expression for $V_{W}$
can be explicitly obtained in terms of $\Gamma$, $\Lambda$ and $\BF$, but is 
algebraic in nature and hence omitted here. 



We now use Theorem~\ref{Theorem:CLT1} to obtain consistency results for 
the eigenvectors obtained from $\widehat{\tilde{\Sigma}} = n^{-1} 
\sum_{i=1}^{n} W^{2} (X_{i}, \BF) \BS (X_{i}; \hat{\mu}_{n})$. 
Suppose that  $\tilde{\Lambda}_{1} > \tilde{\Lambda}_{2} > \ldots > \tilde{\Lambda}_{p}$
are the eigenvalues of $\tilde{\Sigma}$, which we assume are all distinct values. 

\begin{Theorem} \label{Theorem:Eigen1}
Suppose the  spectral decomposition of $\widehat{\tilde{\Sigma}}$ is given 
by $\widehat{\tilde{\Sigma}} = \widehat{\Gamma} \widehat{\tilde{\Lambda}}
\widehat{\Gamma}^T $. Then the matrix of centered and scaled eigenvectors 
$G_{n} = n^{1/2} (\widehat{\Gamma} - \Gamma) $ 
and the vector of centered and scaled eigenvalues 
$L_{n} = n^{1/2} (\widehat{\tilde{\Lambda}} - {\tilde{\Lambda}}) $ have 
independent distributions. The distribution of the random variable $vec~(G_{n})$ converges 
weakly to a $p^2$-variate normal distribution with mean ${\bf 0}_{p^2}$ and the
variance matrix whose $(i, j)$-th block of $p \times p$ entries are given by
\baq
& \sum_{k=1, {}_{k \neq i}}^{p} 
\Bigl[ \tilde{\Lambda}_{i} -  \tilde{\Lambda}_{k} \Bigr]^{-2}
\BE \Bigl[ 
W^4 (Z, \BF_{Z}) \big( 
\BS_{i, k} ({\Lambda}^{1/2} Z; {\bf 0})
\big)^2 
\Bigr]
\bfgamma_k \bfgamma_k^T, 
\text{ if } i = j, \label{equation:DevEq} \\
& - 
\Bigl[ \tilde{\Lambda}_{i} -  \tilde{\Lambda}_{j} \Bigr]^{-2}
\BE \Bigl[ 
W^4 (Z, \BF_{Z}) \big( 
\BS_{i, j} ({\Lambda}^{1/2} Z; {\bf 0})
\big)^2 
\Bigr]
\bfgamma_i \bfgamma_j^T; \text{ if } i \neq j.
\eaq
The distribution of  $L_{n}$ converges 
weakly to a $p$-dimensional normal distribution with mean ${\bf 0}_{p}$ and the
variance-covariance matrix whose $(i, j)$-the element is
\ban
& \BE \Bigl[ 
W^4 (Z, \BF_{Z}) \big( 
\BS_{i, i} ({\Lambda}^{1/2} Z; {\bf 0})
\big)^2 
\Bigr]
- \tilde{\Lambda}_{i}^2, \text{ if } i = j, \\
& \BE \Bigl[ 
W^4 (Z, \BF_{Z}) \big( 
\BS_{i, j} ({\Lambda}^{1/2} Z; {\bf 0})
\big)^2 
\Bigr]
- \tilde{\Lambda}_{i} \tilde{\Lambda}_{j}, \text{ if } i \neq j.
\ean
\end{Theorem}


The proof of this result follows from using Theorem~\ref{Theorem:CLT1} and using 
techniques similar to a corresponding result in \citet{ref:SPL12765_Taskinenetal}, 
and we omit the  algebraic details here.

Recall that the asymptotic variance of the $i$-th eigenvector of the 
\textit{sample covariance matrix}, say $\hat{\bfgamma}_i$ is \citep{ref:AndersonBook09}:
%
\begin{equation} \label{equation:covevEq}
A\BV( \sqrt n\hat \bfgamma_{i}) = 
\sum_{k=1; k \neq i}^p 
\frac{\lambda_i \lambda_k}{(\lambda_i - \lambda_k)^2} \bfgamma_k \bfgamma_k^T; 
\quad 1 \leq i \leq p.
\end{equation}
%
Suppose $\widehat{\tilde{\bfgamma}}_i$ is the $i$-th eigenvector of 
$\widehat{\tilde{\Sigma}}$, whose asymptotic behavior is presented above in 
Theorem~\ref{Theorem:Eigen1}.

This leads to the following useful result:
\begin{Corollary}
The asymptotic relative efficiency of $\widehat{\tilde{\bfgamma}}_i$, 
relative to $\hat{\bfgamma}_i$, is given by 
\ban
& ARE (\widehat{\tilde{\bfgamma}}_i, \hat{\bfgamma}_i; \BF) \\
& =  
\Bigl[ \sum_{k=1; k \neq i}^p \frac{\lambda_i \lambda_k}{(\lambda_i - \lambda_k)^2} \Bigr]
 \Bigl[
\sum_{k=1, {}_{k \neq i}}^{p} 
\biggl[ \tilde{\Lambda}_{i} -  \tilde{\Lambda}_{k} \biggr]^{-2}
\BE \biggl[ 
W^4 (Z, \BF_{Z}) \big( 
\BS_{i, k} ({\Lambda}^{1/2} Z; {\bf 0})
\big)^2 
\biggr]  
\Bigr]^{-1}.
\ean
\end{Corollary}

The proof of this Corollary is immediate.


\section{An Affine Equivariant Robust Measure of Dispersion}
\label{Sec:WSDispersion2}


A desirable invariance property of any dispersion parameter $T_{X}$ corresponding to 
a random variable $X$ is that under affine transformation $Y = AX + b$ the dispersion 
parameter scales to $T_{Y} = A T_{X} A^{T}$. It is clear that $\tilde{\Sigma}$ does 
not possess this property, since it remains unchanged for $X$ and $Y = c X$ for any 
scalar $c > 0$. 

We follow the general framework of M-estimation with data-dependent weights 
\citep{ref:HuberBook81} to construct an affine equivariant counterpart of the 
$\tilde{\Sigma}$. 
Specifically, we implicitly define
\begin{equation} \label{eqn:ADCM}
\Sigma_{*} = \frac{p}{ \BV W (X) } 
\BE \left[ \frac{W^{2}(X) (X - \mu) (X - \mu)^T}
{(X - \mu)^T \Sigma_{*}^{-1}(X - \mu)} \right].
\end{equation}
%

To ensure existence and uniqueness of $\Sigma_{*}$, consider the class of 
dispersion parameters $\Sigma_M$ that are obtained as solutions of the following equation:
%
\begin{equation}
\BE \left[ u( | Z_{M} | )  \frac{Z_{M} Z_{M}^T}{| Z_{M} |^2}  - v( | Z_{M} | ) \BI_p \right] = 0
\end{equation}
%
with $Z_M = \Sigma_M^{-1/2} (X - \mu)$. Under the following assumptions on the scalar valued functions $u$ and $v$, the above equation produces a unique solution \citep{ref:HuberBook81}:
%

\vspace{1em}
\noindent\textbf{(C1)} The function $u(r)/r^2$ is monotone decreasing, 
and $u(r)>0$ for $r > 0$;

\noindent\textbf{(C2)}  The function $v(r)$ is monotone decreasing, and 
$v(r) > 0$ for $r > 0$;

\noindent\textbf{(C3)} Both $u(r)$ and $v(r)$ are bounded and continuous;

\noindent\textbf{(C4)} $u(0) / v(0) < p$;

\noindent\textbf{(C5)} For any hyperplane in the sample space $\mathcal X$, 
(i) $P(H) = \BE \{ \cI_{\{X \in H \}} \} < 1 - p v(\infty) / u(\infty)$ and 
(ii) $P(H) \leq 1/p$.
%

\vspace{1em}
\noindent Putting things into context, in our case we have 
$v (\cdot) = p^{-1}\BV W (X)$, 
$u (\cdot) = W^{2}(X)$. 
%We use the notation $\BF_{|Z|}$ for the cumulative distribution function of $|Z|$.
We proceed to verify the other conditions for the weight functions 
$W_{HSD} (\cdot)$, 
$W_{MhD} (\cdot)$ and $W_{PD} (\cdot)$ discussed earlier.

It is easy to verify that the resulting $u (\cdot)$ from the above choices 
satisfy (C1) and (C3). 
Note that $v (\cdot)$ is a finite positive constant, 
and (C2) and (C3) are also easily satisfied. 
Since $u (0) = 0$ in all the above cases, (C4) is also easy to check. Since $X$ is 
absolutely continuous, (C5) holds trivially.



In order to compute the sample version of $\Sigma_{*}$, we solve (\ref{eqn:ADCM}) 
iteratively by obtaining a sequence of positive definite matrices 
$\hat{\Sigma}^{(k)}_{*}$ until convergence. Thus, using the location 
estimator $\hat{\mu}_{n}$, we may iterate
%
\ban
\hat{\Sigma}^{(k+1)}_{*}  = 
\frac{p}{ \BV W (X) } 
\BE \left[ \frac{W^{2}(X) (X - \hat{\mu}_{n}) (X - \hat{\mu}_{n})^T}
{(X - \hat{\mu}_{n})^T (\hat{\Sigma}^{(k)}_{*})^{-1}(X - \hat{\mu}_{n})} \right].
\ean
%

The asymptotic properties of $\hat{\Sigma}_{*}$ can be obtained using methods similar 
to those of Section~\ref{Sec:WSDispersion1}, and techniques presented in 
\cite{ref:Biometrika00603_CrouxHaesbroeck} and elsewhere. We state the following result and omit its proof.

\begin{Theorem}
\label{Thm:Eigen2}
The asymptotic covariance matrix of an eigenvector of the sample 
affine equivariant scatter functional $\hat{\Sigma}_{*}$ is given by
\ban 
V_{12}
\sum_{k=1, k \neq i}^p \frac{\lambda_i \lambda_k}{\lambda_i - \lambda_k} 
\bfgamma_i \bfgamma_k^T,
\ean
where $V_{12}$  is the asymptotic variance of an off-diagonal element of 
$\hat{\Sigma}_{*}$ when the underlying distribution is $\BF_{Z}$. 
It follows that if $\hat{\bfgamma}_{*, i}$ is the $i$-th eigenvector of $\hat{\Sigma}_{*}$,
%
\begin{equation}
ARE (\hat{\bfgamma}_{*, i}, \hat\bfgamma_{i}; \BF) = V_{12}^{-1} = 
\frac{\left[ \BE ( p u (| Z |)  + u'( | Z |) | Z | ) \right]^2}
{p^2 (p+2)^2 \BE (u (| Z |)^2 \BE (\BS_{12} (Z; {\bf 0}))^2}.
\end{equation}
%
\end{Theorem}


\section{Robust Estimation of Eigenvalues, and a Plug-in Estimator of $\Sigma$}
\label{Sec:Eigen}


As seen in Theorem~\ref{Thm:WSVariance}, eigenvalues of the $\tilde{\Sigma}$ 
are not same as the population eigenvalues. In this section, we discuss on robust 
estimation of $\lambda_{i}$'s using $\tilde{\Sigma}$. Assume the data 
is centered, the robust estimator from Section~\ref{Sec:WSQuantiles} suffices. 
We start by computing 
the sample version $\widehat{\tilde{\Sigma}}$ and its spectral decomposition:
\ban 
\widehat{\tilde{\Sigma}} = \widehat{{\Gamma}} \widehat{\tilde{\Lambda}}
\widehat{{\Gamma}}^{T}
\ean 
We then use the following steps:

\begin{enumerate}
\item Randomly divide the sample indices $\{1,2, \ldots, n\}$ into $k$ disjoint groups $\{G_1,\ldots, G_k \}$ of size $\lfloor n/k \rfloor$ each.

\item Transform the data matrix: 
$\bfS = \widehat{{\Gamma}}^{T} \bfX$.

\item Calculate coordinate-wise variances for each group of indices $G_j$:
%
\ban
\lambda_{i, j}^{\dagger} = \frac{1}{|G_j|} \sum_{l \in G_j} \bigl( 
S_{l i} - \bar{S}_{G_{j}, i} \bigr)^2; \quad i = 1, \ldots, p; \  j = 1, \ldots, k.
\ean
where $\bar{\bfS}_{G_j} = (\bar{S}_{G_j, 1}, \ldots, \bar{S}_{G_j,p})^T$ is the vector of 
column-wise means of $\bfS_{G_j}$, the submatrix of $\bfS$ with row indices in $G_j$.
%

\item Obtain estimates of eigenvalues by taking coordinate-wise medians of these variances:
%
\ban
{\lambda}^{\dagger}_{i} = \text{median} (\lambda_{i,1}^{\dagger}, 
\ldots , \lambda_{i,k}^{\dagger} ); \quad 
i = 1, \ldots, p.
\ean
%
\end{enumerate}
%







We collect ${\lambda}^{\dagger}_{i}$, $ i =1, \ldots, p$ in the diagonal matrix 
${\Lambda}^{\dagger} = \diag ({\lambda}^{\dagger}_{1}, \ldots, 
{\lambda}^{\dagger}_{p})$.
The number of subgroups used to calculate this median-of-small-variances estimator can 
be determined following \cite{ref:Bernoulli152308_Minsker_Median_Banach}. 
There can be other ways of estimating 
the eigenvalues of $\Sigma$ using $\bfS$ also, we will pursue such methods elsewhere.
We construct a consistent 
plug-in estimator of the population covariance matrix $\Sigma$ as 
\ban 
{\Sigma}^{\dagger}
= \widehat{{\Gamma}} {\Lambda}^{\dagger} \widehat{{\Gamma}}^{T}.
\ean
Let $| A |_{F}$ denote the Frobenius norm of a matrix $A$, in other words,
$| A |_{F} = (\text{trace} A^{T} A)^{1/2}$.
The following result establishes that this is a consistent estimator of $\Sigma$:

\begin{Theorem}\label{Thm:pluginSigma}
Suppose that as $n \rightarrow \infty$, $k \rightarrow \infty$ and 
$n/k \rightarrow \infty$.
Then we have
%
\ban
\| {\Sigma}^{\dagger} - \Sigma \|_F \stackrel{P}{\rightarrow} 0.
\ean
%
\end{Theorem}

\begin{proof}[Proof of Theorem \ref{Thm:pluginSigma}]
This proof has many algebraic steps, and we sketch the main arguments below.
Suppose $\hat{A} = \widehat{{\Gamma}}^{T}  \Sigma \widehat{{\Gamma}}$. 

Owing to the fact that the Frobenius norm is invariant under rotations and 
that $p$ is finite and fixed, it suffices to show that the off-diagonal elements of 
$\hat{A}$ converge in probability to zero, and that the difference between 
the $i$-th diagonal element of $\hat{A}$ and ${\lambda}^{\dagger}_{i}$ converges to 
zero for any $i = 1, \ldots, p$.

Now notice that from Theorem~\ref{Theorem:Eigen1} we have that $\widehat{{\Gamma}} 
= {{\Gamma}} + R_{n1}$, where the $(i, j)$-th element of the remainder 
$R_{n1, i, j}$ satisfies $\BE  R_{n1, i, j}^{2} = O(n^{-1})$. 

We can show, using standard algebra, that 
\ban
\hat{A} = \Lambda + R_{n2}, 
\ean
where the $(i, j)$-th element of the remainder 
$R_{n2, i, j}$ satisfies $\BE  R_{n2, i, j}^{2} = O(n^{-1})$. 
This follows immediately from above, the fact that $p$ is finite and fixed, and all 
elements of $\Lambda$ are constants. This immediately establishes the case for the 
off-diagonal elements. 

For the diagonal elements, notice that since $k \rightarrow \infty$, each 
coordinate-wise variance $\lambda_{i, j}^{\dagger}$ for each group of indices $G_j$ 
is a consistent estimator of $\lambda_{i}$. The result follows.

\end{proof}


\section{Influence Functions of Dispersion Measures}
\label{Sec:RE_Dispersion}


We retain the framework adopted in Section~\ref{Sec:WSDispersion1}, and discuss in this 
section the robustness and efficiency properties associated with $\tilde{\Sigma}$ 
and $\Sigma_{*}$, and principal components derived therefrom. 
We do not discuss $\Sigma^{\dagger}$ here, since  the 
properties of that approach follow from those of  $\tilde{\Sigma}$. 
We additionally assume that the eigenvalues of $\Sigma$ are 
distinct, and given by  $\lambda_1 > \lambda_2 > \ldots > \lambda_p$, to avoid 
several additional technical conditions for the theoretical results to follow. 
The case where the eigenvalues of $\Sigma$ can have multiplicity greater than one 
requires no additional conceptual development, but does require considerable algebraic 
manipulations.

For studying 
the robustness aspect, we first present some results relating to influence functions 
in the current context. 
 Influence functions quantify how much influence a sample point, especially 
an infinitesimal contamination, has on any functional of a probability 
distribution \citep{ref:HampelBook86}. Given any probability distribution 
$\BH \in \cM$, the influence function of any point $x_0 \in \mathcal{X}$ for 
some functional $T(\BH)$ on the distribution is defined as
%
\ban
IF(x_0; T,\BH) =
\lim_{\epsilon \rightarrow 0} \frac{1}{\epsilon} (T(\BH_\epsilon) - T(\BH)),
\ean
%
where 
$\BH_\epsilon = (1-\epsilon)\BH + \epsilon \delta_{x_0}$; $\delta_{x_0}$ being 
the distribution with point mass at $x_0$. When $T(\BH) = E_\BH f$ for some 
$\BH$-integrable function $f$, $IF(x_0; T,\BH) = f(x_0) - T(\BH)$.

It now follows that
%
\ban
IF (x_0; \tilde{\Sigma}, \BF) =
W^{2} (x_{0}) \BS(x_0; \mu) - \tilde \Sigma.
\ean
%
Recall that $\tilde{\lambda}_{1} > \tilde{\lambda}_{2} > \ldots > \tilde{\lambda}_{p}$
are the eigenvalues of $\tilde{\Sigma}$, which we assume are all distinct values. 

\begin{Proposition}\label{Thm:IF}
The influence function of $\bfgamma_{i}$ as follows: 
\ban
IF(x_0; \bfgamma_{i}, \BF)  & = 
\sum_{k = 1; k \neq i}^p \frac{1}{\tilde{\lambda}_{i} - \tilde{\lambda}_{k}} 
\left\{ \bfgamma^T_k IF(x_0; \tilde \Sigma,\BF)  \bfgamma_{i} \right\} 
\bfgamma_k \notag \\
& =  \sum_{k=1; k \neq i}^p \frac{1}{\tilde{\lambda}_{i} - \tilde{\lambda}_{k}}
W^{2} (x_{0}) \left\{  \bfgamma^T_{k} \BS (x_0; \mu) \bfgamma_{i} \right\} 
\bfgamma_{k}.
\ean
\end{Proposition}

The proof of Proposition~\ref{Thm:IF} follows from \cite{ref:JRSSB79217_Sibson} and 
\cite{ref:Biometrika00603_CrouxHaesbroeck}, we omit the details. 

If the weight function $W (\cdot)$ is a bounded function, as is the case of $W_{HSD}$, 
$W_{MhD}$, and $W_{PD}$, 
the influence function given in  Proposition~\ref{Thm:IF} is bounded, indicating good 
robustness properties of the principal component analysis.



We now derive the influence function for $\Sigma_{*}$. 
\begin{Proposition}\label{Thm:IF2}
The influence function of $\Sigma_{*}$ is given by
\ban 
IF ( x_{0}, \Sigma_{*}, \BF) = 
\alpha_{\Sigma_{*}} ( | x_{0} |; \BF_{Z} ) 
\BS( x_0; \mu) 
- \beta_{\Sigma_{*}} (| x_{0} |; \BF_{Z} ) \Sigma_{*}.
\ean
\end{Proposition}

\begin{proof}[Proof of Proposition~\ref{Thm:IF2}:]

Let $z_{0} = \Lambda^{-1/2} \Gamma^T  (x_0 - \mu) = (z_{0 1}, \ldots, z_{0 p})^T$.
As a first step, since $\Sigma_{*}$ is affine equivariant, we obtain from 
\cite{ref:Biometrika00603_CrouxHaesbroeck} 
that 
\ban 
IF ( x_{0}, \Sigma_{*}, \BF) = \Sigma_{*}^{1/2} 
IF ( z_{0}, \Sigma_{*}, \BF_{Z})  \Sigma_{*}^{1/2}.
\ean
From Lemma1 of \citep{ref:HampelBook86}, page 276, we obtain that there exist 
scalar valued functions $\alpha_{\Sigma_{*}} ( | x_{0} |; \BF_{Z} )$ and 
$\beta_{\Sigma_{*}} ( | x_{0} |; \BF_{Z} ) $ such  that 
\ban
IF ( z_{0}, \Sigma_{*}, \BF_{Z}) = \alpha_{\Sigma_{*}} ( | x_{0} |; \BF_{Z} ) 
\BS( z_0; {\mathbf 0}) 
- \beta_{\Sigma_{*}} (| x_{0} |; \BF_{Z} ) \BI_{p}, 
\ean
consequently we obtain
\ban 
IF ( x_{0}, \Sigma_{*}, \BF) = 
\alpha_{\Sigma_{*}} ( | x_{0} |; \BF_{Z} ) 
\BS( x_0; \mu) 
- \beta_{\Sigma_{*}} (| x_{0} |; \BF_{Z} ) \Sigma_{*}.
\ean
\end{proof}

Suppose ${\lambda}_{* 1} > {\lambda}_{* 2} > \ldots > {\lambda}_{* p}$
are the eigenvalues of ${\Sigma}_{*}$, which we assume are all distinct values. 
Also denote the $i$-th eigenvector of $\Sigma_{*}$ by 
$\bfgamma_{* i} = (\gamma_{* i 1}, \ldots, \gamma_{* i p})^T$ for $1 \leq i \leq p$.

\begin{Proposition}\label{Thm:IF3}
The influence function of $\bfgamma_{* i} $ may be obtained as 
\ban
IF(x_0; \bfgamma_{*i}, \BF)  & = 
\sum_{k = 1; k \neq i}^p \frac{1}{{\lambda}_{*i} - {\lambda}_{*k}} 
\left\{ \bfgamma^T_{* k} IF(x_0; \tilde \Sigma,\BF)  \bfgamma_{* i} \right\} 
\bfgamma_{* k} \notag \\
& =  
\alpha_{\Sigma_{*}} ( | x_{0} |; \BF_{Z} ) \sum_{k=1; k \neq i}^p 
\frac{1}{{\lambda}_{*i} - {\lambda}_{*k}}
\left\{ \bfgamma^T_{* k} \BS (x_0; \mu) \bfgamma_{* i} \right\} 
\bfgamma_{* k}.
\ean
\end{Proposition}

We omit the proof of Proposition~\ref{Thm:IF3}, which follows along similar lines to 
to rest of the computations of this section.
It can be shown that when $W (\cdot)$ is a bounded function, 
$\alpha_{\Sigma_{*}} ( | x_{0} |; \BF_{Z} )$ is also bounded, along the lines of 
\cite{ref:HuberBook81}, which in turn implies that the influence function for a 
principal component based on $\Sigma_{*}$ is also bounded.



\section{Simulation Studies}
\label{Sec:Simulation}

We report a number of numerical simulation studies on several properties relating to 
$\tilde{\Sigma}$ and  $\Sigma_{*}$, and their eigenvalues and eigenvectors, 
on datasets with or without influential points, to illustrate the finite sample 
efficiency and robustness properties of the proposed weighted estimators. We compare 
these proposed estimators with techniques that exists in literature, specifically, 
the Sign Covariance Matrix (SCM) and Tyler's shape matrix \citep{ref:AoS87234_Tyler}.

\subsection{Efficiency of different robust estimators}

We compare the performance of $\tilde{\Sigma}$ and $\Sigma_{*}$ with that of the 
SCM and Tyler's scatter matrix. For this study, we fix the dimension $p = 4$.
We consider six elliptical distributions, 
and from every distribution draw 10000 samples each for sample sizes $n = 20, 50, 100, 
300$ and $500$. All distributions are centered at ${\bf 0}_p$, and have covariance matrix 
$\Sigma = \diag(4, 3, 2, 1)$. 

We use the concept of principal angles 
\citep{ref:LinearAlgebraApplications9281_MiaoBenIsrael} 
to find out error estimates for the 
first eigenvector of a scatter matrix. In our case, the first eigenvector is
%
\ban
\bfgamma_1 = (1, \overbrace{0, \ldots, 0}^{p - 1})^T.
\ean
%
We measure the prediction error for an eigenvector estimator (say, $\tilde\bfgamma_{1}$), 
using the smallest angle between the true and predicted vectors, i.e. 
$ \cos^{-1} | \tilde\bfgamma_{1}^T \hat\bfgamma_1 | $. A small absolute value of this 
angle means to better prediction. We repeat this 10,000 times and calculate the 
\textbf{Mean Squared Prediction Angle}:
%
\ban
MSPA(\hat \bfgamma_{1}) =
\frac{1}{10000} \sum_{m=1}^{10000} \left( \cos^{-1} 
\left|\bfgamma_1^T \tilde\bfgamma^{(m)}_{1} \right| \right)^2.
\ean
%
where $\tilde\bfgamma^{(m)}_{1}$ is the value of $\tilde\bfgamma_{1}$ in the 
$m$-th replication, $m = 1, \ldots, 10,000$. 
The finite sample efficiency of $\tilde\bfgamma_{1}$ relative to that 
from the sample covariance matrix, i.e. $\hat\bfgamma_{1}$ is obtained as:
\ban
 FSE ( \hat\bfgamma_{1}, \hat\bfgamma_{1}) = 
 \frac{ MSPA(\hat\bfgamma_{0, 1})}{MSPA(\hat\bfgamma_{1})}.
 \ean

The results from this simulation exercise are presented in Table~\ref{table:FSEtable4}.
It can be seem that  $\tilde{\Sigma}$-based estimators (columns 3-5) 
outperform SCM and Tyler's $M$-estimator of scatter. Among the 3 depth functions 
considered, projection depth gives the best results. Its finite sample performances are 
better than Tyler's and Huber's M-estimators of scatter, as well as their symmetrized 
counterparts that are much more computationally intensive (see Table 4 in 
\cite{ref:JMVA071611_Sirkiaetal}). The affine equivariant ${\Sigma}_{*}$-based estimators
 (columns 6-8) are even more efficient.

\begin{table}[t]
\begin{scriptsize}
    \begin{tabular}{c|cc|ccc|ccc}
    \hline
    4-variate $t_5$    & SCM  & Tyler & $\tilde{\Sigma}$-H & $\tilde{\Sigma}$-M & $\tilde{\Sigma}$-P & ${\Sigma}_{*}$-H & ${\Sigma}_{*}$-M & ${\Sigma}_{*}$-P \\ \hline
    $n$=20             & 1.04 & 1.02  & 1.10   & 1.07   & 1.02  & 1.09    & 1.07    & 0.98   \\
    $n$=50             & 1.08 & 1.08  & 1.16   & 1.16   & 1.13  & 1.19    & 1.19    & 1.13   \\
    $n$=100            & 1.31 & 1.31  & 1.42   & 1.38   & 1.36  & 1.46    & 1.44    & 1.36   \\
    $n$=300            & 1.46 & 1.54  & 1.81   & 1.76   & 1.95  & 1.88    & 1.88    & 1.95   \\
    $n$=500            & 1.92 & 1.93  & 2.23   & 2.03   & 2.31  & 2.35    & 2.19    & 2.39   \\ \hline
    4-variate $t_6$     & SCM  & Tyler & $\tilde{\Sigma}$-H & $\tilde{\Sigma}$-M & $\tilde{\Sigma}$-P & ${\Sigma}_{*}$-H & ${\Sigma}_{*}$-M & ${\Sigma}_{*}$-P \\ \hline
    $n$=20             & 1.00 & 1.05  & 1.03   & 1.05   & 1.00  & 1.04    & 1.04    & 0.95   \\
    $n$=50             & 1.03 & 1.01  & 1.13   & 1.12   & 1.11  & 1.19    & 1.17    & 1.10   \\
    $n$=100            & 1.08 & 1.12  & 1.25   & 1.23   & 1.27  & 1.24    & 1.25    & 1.22   \\
    $n$=300            & 1.34 & 1.36  & 1.64   & 1.52   & 1.60  & 1.67    & 1.61    & 1.68   \\
    $n$=500            & 1.26 & 1.34  & 1.55   & 1.49   & 1.60  & 1.65    & 1.61    & 1.69   \\ \hline
    4-variate $t_{10}$ & SCM  & Tyler & $\tilde{\Sigma}$-H & $\tilde{\Sigma}$-M & $\tilde{\Sigma}$-P & ${\Sigma}_{*}$-H & ${\Sigma}_{*}$-M & ${\Sigma}_{*}$-P \\ \hline
    $n$=20             & 0.90 & 0.89  & 0.95   & 0.98   & 0.98  & 0.96    & 1.01    & 0.95   \\
    $n$=50             & 0.90 & 0.91  & 1.01   & 0.98   & 0.98  & 1.03    & 1.04    & 0.99   \\
    $n$=100            & 0.87 & 0.87  & 0.93   & 0.95   & 1.01  & 0.99    & 1.01    & 1.05   \\
    $n$=300            & 0.87 & 0.87  & 1.09   & 1.09   & 1.17  & 1.14    & 1.16    & 1.23   \\
    $n$=500            & 0.88 & 0.92  & 1.10   & 1.10   & 1.23  & 1.19    & 1.22    & 1.29   \\ \hline
    4-variate $t_{15}$  & SCM  & Tyler & $\tilde{\Sigma}$-H & $\tilde{\Sigma}$-M & $\tilde{\Sigma}$-P & ${\Sigma}_{*}$-H & ${\Sigma}_{*}$-M & ${\Sigma}_{*}$-P \\ \hline
    $n$=20             & 0.92 & 0.90  & 0.94   & 0.94   & 0.96  & 0.95    & 0.97    & 0.89   \\
    $n$=50             & 0.82 & 0.83  & 0.88   & 0.91   & 0.93  & 0.88    & 0.93    & 0.93   \\
    $n$=100            & 0.84 & 0.87  & 0.92   & 0.95   & 1.00  & 0.93    & 0.96    & 1.00   \\
    $n$=300            & 0.73 & 0.75  & 0.96   & 0.99   & 1.10  & 1.00    & 1.06    & 1.12   \\
    $n$=500            & 0.73 & 0.76  & 0.95   & 0.96   & 1.06  & 0.94    & 0.97    & 1.06   \\ \hline
    4-variate $t_{25}$  & SCM  & Tyler & $\tilde{\Sigma}$-H & $\tilde{\Sigma}$-M & $\tilde{\Sigma}$-P & ${\Sigma}_{*}$-H & ${\Sigma}_{*}$-M & ${\Sigma}_{*}$-P \\ \hline
    $n$=20             & 0.89 & 0.92  & 0.92   & 0.92   & 0.90  & 0.96    & 0.95    & 0.89   \\
    $n$=50             & 0.82 & 0.84  & 0.89   & 0.90   & 0.91  & 0.93    & 0.96    & 0.92   \\
    $n$=100            & 0.77 & 0.76  & 0.90   & 0.90   & 0.96  & 0.94    & 0.98    & 1.04   \\
    $n$=300            & 0.73 & 0.77  & 0.93   & 0.91   & 0.98  & 1.00    & 0.98    & 1.03   \\
    $n$=500            & 0.67 & 0.71  & 0.83   & 0.83   & 0.96  & 0.88    & 0.90    & 1.00   \\ \hline
    4-variate Normal   & SCM  & Tyler & $\tilde{\Sigma}$-H & $\tilde{\Sigma}$-M & $\tilde{\Sigma}$-P & ${\Sigma}_{*}$-H & ${\Sigma}_{*}$-M & ${\Sigma}_{*}$-P \\ \hline
    $n$=20             & 0.82 & 0.84  & 0.87   & 0.90   & 0.91  & 0.89    & 0.93    & 0.89   \\
    $n$=50             & 0.80 & 0.81  & 0.87   & 0.88   & 0.88  & 0.88    & 0.92    & 0.88   \\
    $n$=100            & 0.68 & 0.71  & 0.80   & 0.85   & 0.91  & 0.82    & 0.86    & 0.92   \\
    $n$=300            & 0.61 & 0.63  & 0.82   & 0.85   & 0.93  & 0.86    & 0.91    & 0.96   \\
    $n$=500            & 0.60 & 0.64  & 0.77   & 0.80   & 0.90  & 0.82    & 0.86    & 0.96   \\ \hline
    \end{tabular}
\end{scriptsize}
\caption{Finite sample efficiencies of estimators of the first eigenvector based on 
several scatter matrices in dimension $p=4$. The notation
 H, M or P after $\tilde{\Sigma}$ or ${\Sigma}_{*}$ indicates the depth function 
 used for the weights: H = halfspace depth, M = Mahalanobis depth, P = projection depth.}
\label{table:FSEtable4}
\end{table}





\subsection{Influence function comparison}

%\begin{comment}

\begin{figure}[]
	\centering
		\includegraphics[width=0.8\textwidth]{./Plots/IFnorm.png}
	\caption{Plot of the norm of influence function for first eigenvector of (a) sample 
	covariance matrix, (b) SCM, (c) Tyler's scatter matrix and $\tilde{\Sigma}$ for 
	weights obtained from (d) Halfspace depth, (e) Mahalanobis depth, (f) Projection depth 
	for a bivariate normal distribution with $\bfmu = {\bf 0}, \Sigma = \diag(2,1)$}
	\label{fig:IFnorm}
\end{figure}


In Figure \ref{fig:IFnorm} we consider first eigenvectors of $\tilde{\Sigma}$, 
the Sign Covariance Matrix (SCM) 
and Tyler's shape matrix \citep{ref:AoS87234_Tyler}. We generate data from and set
$\BF \equiv \mathcal{N}_2({ 0}, \diag(2,1))$ and plot norms of the 
eigenvector influence functions for different values of $x_0$. 
Let us denote the $i$-th eigenvector of  the Sign Covariance Matrix and Tyler's shape 
matrix by  $\gamma_{S,i}$ and $\gamma_{T,i}$, respectively. 
Their influence functions are given  as follows:
%
\ban
 IF(x_0; \gamma_{S,i}, \BF) & = 
\sum_{k=1; k \neq i}^p \frac{\BS_{ik}(\Lambda^{1/2} z_0, { 0})}{\lambda_{S,i} - \lambda_{S,k}} \gamma_k; \\
%\quad 
& \text{ where }
\lambda_{S,i} = \BE_Z \left( \frac{\lambda_i z_i^2}{\sum_{j=1}^p \lambda_j z_j^2} \right),\\
IF(x_0; \gamma_{T,i}, \BF) & =  (p+2) \sum_{k=1; k \neq i}^p \frac{\sqrt{\lambda_i 
\lambda_k}}{\lambda_i - \lambda_k}\BS_{ik}(z_0; {0}) \gamma_k. 
%\label{eqn:IFeqnTyler}
\ean
%
Panels (b) and (c) in Figure~\ref{fig:IFnorm}, corresponding to  Sign Covariance Matrix 
and Tyler's shape matrix respectively, exhibit an `inlier effect', that is, 
points close to the center  having high influence,  which results in loss of efficiency. 
On the other hand, the influence function for eigenvector estimates of the 
sample covariance matrix  (panel (a)) is 
unbounded and makes the corresponding estimates non-robust. In comparison, the 
$\tilde{\Sigma}$ corresponding to weights derived from projection depth, 
half-space depth and Mahalanobis depth  have bounded influence functions {\it and} 
small values of the influence function at `deep' points.

\subsection{Efficiency of affine equivariant robust estimator}

We study the finite sample efficiency properties of $\Sigma_{*}$ 
using a simulation exercise. 
We consider 6 different elliptic distributions,  namely, the $p$-variate multivariate 
Normal distribution and the multivariate $t$ distributions corresponding to degrees 
of freedom 5, 6, 10, 15 and 25. We compute the ARE of the estimator for the 
first eigenvector using $\Sigma_{*}$, using weights based on the projection depth 
(PD) and the halfspace depth (HSD), thus this simulation is an illustration of how 
different choices of weights affect the results in the context of 
Theorem~\ref{Thm:Eigen2}.  We consider using the sample covariance 
matrix as the baseline method for this study. The ARE values are computed by
using Monte-Carlo simulation of $10^6$ samples and subsequent numerical integration.
We report the results of this exercise in Table~\ref{table:AREtable}.  
Based on these results, we notice that  $\Sigma_{*}$  is particularly efficient in lower 
dimensions for distributions with heavier tails ($t_5$ and $t_6$), while for distributions 
with lighter tails, the AREs increase with data dimension. At higher values of $p$,
note that $\Sigma_{*}$ is almost as efficient as the sample covarnace matrix even
 when the data comes from multivariate normal distribution.

\begin{table}[t]
\centering
\begin{footnotesize}
\begin{tabular}{c|cccc|cccc}
    \hline
    & \multicolumn{4}{c|}{PD} & \multicolumn{4}{c}{HSD} \\\cline{2-9}
    Distribution & $p=2$  & $p=5$  & $p=10$ & $p=20$ & $p=2$  & $p=5$  & $p=10$ & $p=20$ \\ \hline
    $t_5$           & 4.73 & 3.99 & 3.46 & 3.26 & 4.18 & 3.63 & 3.36 & 3.15 \\
    $t_6$           & 2.97 & 3.28 & 2.49 & 2.36 & 2.59 & 2.45 & 2.37 & 2.32 \\
    $t_{10}$          & 1.45 & 1.47 & 1.49 & 1.52 & 1.30 & 1.37 & 1.43 & 1.49 \\
    $t_{15}$          & 1.15 & 1.19 & 1.23 & 1.27 & 1.01 & 1.10 & 1.17 & 1.24 \\
    $t_{25}$          & 0.97 & 1.02 & 1.07 & 1.11 & 0.85 & 0.94 & 1.02 & 1.08 \\
    MVN          & 0.77 & 0.84 & 0.89 & 0.93 & 0.68 & 0.77 & 0.84 & 0.91 \\ 
    \hline
\end{tabular}
\end{footnotesize}
\caption{Table of AREs of the estimator for the first eigenvector estimation using 
$\Sigma_{*}$, relative to using the sample covariance matrix, for different choices of 
dimension $p$. The data-generating distributions are the multivariate Normal (MVN), 
and multivariate $t$-distributions with degrees of freedom 5, 6, 10, 15 and 25. Weights 
for $\Sigma_{*}$ are based on either the projection depth (PD) or the half-space 
depth (HSD).}
\label{table:AREtable}
\end{table}

%\end{comment}



\subsection{Robust sufficient dimension reduction and supervised learning}

One of the main usages of obtaining dispersion estimators and their eigenvalues and 
eigenvectors is in \textit{dimension reduction} techniques. Examples of such uses are in 
\textit{principal component regression, partial least squares} and \textit{envelope 
methods}. We illustrate below the latter technique, in the context of 
\textit{sufficient dimension reduction} (SDR). For details on envelope methods and other 
uses of robust estimators of dispersion and eigen-structures, see 
\citet{ref:Sinica10927_Cooketal, ref:PhilTransRoyalSoc094385_AdragniCook, 
ref:JASA15599_CookZhang} 
and references and citations of these 
articles. In the context of multivariate-response ($Y_{i} \in \BR^{q}$) linear regression, 
the envelope method proposes the model $Y_{i} = \alpha + \Gamma_{1} \eta x_{i} + e_{i}$, 
where $e_{i}$ are independent mean zero Gaussian  noise terms with covariance matrix 
$\Sigma$ whose spectral representation can be written as 
\ban 
\Sigma = \Gamma \Lambda \Gamma^{T}
& = 
\left( \begin{array}{ll}
\Gamma_{0} & \Gamma_{1} 
\end{array}
\right)
\left( \begin{array}{ll}
\Lambda_{0} & 0 \\
0 & \Lambda_{1} 
\end{array}
\right)
\left( \begin{array}{l}
\Gamma_{0} \\
\Gamma_{1} 
\end{array}
\right) \\
& = \Gamma_{0} \Lambda_{0} \Gamma_{0}^{T}
+ \Gamma_{1} \Lambda_{1} \Gamma_{1}^{T}.
\ean
Thus, the eigenvectors of $\Sigma$ are partitioned into two blocks:  
$\Gamma_{1} \in \BR^{q} \times \BR^{d}$ and 
$\Gamma_{0} \in \BR^{q} \times \BR^{q-d}$, and 
the regression coefficient of $Y_{i}$ on $x_{i}$  is given by 
$\Gamma_{1} \eta$ for some $\eta \in \BR^{d} \times \BR^{p}$. 
Dimension reduction is achieved  when $d \ll p$, typically without extraneous 
assumptions like sparsity. The envelope model for generalized linear models is 
discussed in \citet{ref:PhilTransRoyalSoc094385_AdragniCook}, 
and may be used for supervised 
learning. Nonlinear regression models may also be handled similarly.

Given a set of examples $\{ (Y_{i}, X_{i}), i = 1, \ldots, n \}$, an envelope-based 
prediction for the response $Y$ for any $X$ may be obtained from 
\ban
\hat{Y} (X) & = \bigl[ \sum_{i = 1}^{n} w_{i} \bigr]^{-1}
\sum_{i = 1}^{n} w_{i} Y_{i}, 
\text{ where } \\ 
 w_{i} &= \exp \left[ -\frac{1}{\hat\sigma^2}  | \hat\Gamma_{1}^T (X - X_i) |^2 \right].
\ean
The above assumes that the covariates come from the Gaussian distribution 
$N_{p} ({\bf 0}_{p}, \sigma^{2} \BI_{p})$, and appropriate 
changes may be made for other distributions. 

We design a robust version of the above, by using weighted spatial medians for location 
parameters corresponding to the distributions of $X$ and $X | Y$, and using the first $d$ 
eigenvectors of $\tilde{\Sigma}$ as $\hat\Gamma_{1}$. A robust location estimator for 
the distribution of $X | Y$ is required for the estimation of $\sigma^2$. 
Details are available in \cite{ref:PhilTransRoyalSoc094385_AdragniCook}.

Ina non-linear regression model, we compare the performance of the robust version of 
SDR with the original method of 
\cite{ref:PhilTransRoyalSoc094385_AdragniCook} 
with or without the presence of bad leverage points in $\Sigma$. 
For any given choice of covariate dimension $p$, we take $n=200$ and $d=1$, 
and generate the responses 
$Y_1, \ldots, Y_n$ as independent standard normal, and 
$X | Y$ as Normal with mean $Y + Y^{2} + Y^3$ in each of the $p$ coordinates, 
and variance $25 \BI_p$.
We measure performances of the SDR models by their mean squared prediction error on 
another set of 200 observations  generated similarly, and taking the average of these 
errors on 100 such training-test pairs of datasets. The above steps 
are repeated for the choices of $p = 5, 10, 25, 50, 75, 100, 125, 150$.

%\begin{comment}
\begin{figure}[t]
%\captionsetup{justification=centering, font=footnotesize}
\begin{center}
\begin{tabular}{ll}
\includegraphics[width=0.4\textwidth]{./Plots/SDRcomparison_noout} &
\includegraphics[width=0.4\textwidth]{./Plots/SDRcomparison_out}
\end{tabular}
\caption{Average prediction errors for two methods of SDR (a) in absence and (b) in presence of outliers}
\label{fig:SDRfig}
\end{center}
\end{figure}
%\end{comment}

Panel (a) of figure \ref{fig:SDRfig} compares prediction errors using both robust and 
maximum likelihood SDR estimates when the covariates contain no outliers: here the two 
methods are virtually indistinguishable. We then introduce outliers in each of the 100 
datasets by adding 100 to first $p/5$ coordinates of the first 10 observed covariate 
values, and repeat the analysis. Panel (b) of the figure shows that the robust SDR method 
remains more accurate in predicting out of sample observations for all values of $p$ than 
the standard SDR.




\input{WeightedSign_EJS_DA_02_24_19_1}
\section{Conclusions}
\label{Sec:Conclusion}

We propose the use of a weighted multivariate sign transformation for robust 
estimation and inference, and as demonstrated by theoretical results and several 
simulation studies and data examples, in many situations using a data-depth driven weight 
function leads to considerable efficiency gain without compromising robustness 
properties. Our methodology seems to suggest new ways of identifying 
El-Ni\~no or La-Ni\~na events from the sea-surface temperature anomaly data, 
which will be studied further later.

Several of our results stated above are for data from the Euclidean space $\BR^{p}$, where 
$p$ is fixed. The cases where $p$ increases with sample size and may be higher than sample 
size, and where data are from a separable Hilbert space, will be considered in a future 
work. There are few conceptual difficulties to such extensions, however, there are 
several technical and algebraic challenges. 



\section*{Acknowledgements}
The research of SC is partially  supported by the National Science Foundation (NSF) 
under grants 
\#~DMS-1622483, \#~DMS-1737918, \#~OAC-1939916 and \#~DMR-1939956.



\bibliographystyle{apalike}
\bibliography{WeightedSignBib_09_17_20}

%
%\section{Ordinary text}
%
%The ends  of words and sentences are marked
%  by   spaces. It  doesn't matter how many
%spaces    you type; one is as good as 100.  The
%end of   a line counts as a space.
%
%One   or more   blank lines denote the  end
%of  a paragraph.
%
%Since any number of consecutive spaces are treated like a single
%one, the formatting of the input file makes no difference to
%      \TeX,         % The \TeX command generates the TeX logo.
%but it makes a difference to you.
%When you use
%      \LaTeX,       % The \LaTeX command generates the LaTeX logo.
%making your input file as easy to read as possible
%will be a great help as you write your document and when you
%change it.  This sample file shows how you can add comments to
%your own input file.
%
%Because printing is different from typewriting, there are a
%number of things that you have to do differently when preparing
%an input file than if you were just typing the document directly.
%Quotation marks like
%       ``this''
%have to be handled specially, as do quotes within quotes:
%       ``\,`this'                  % \, separates the double and single quote.
%    is what I just
%    wrote, not  `that'\,''.
%
%Dashes come in three sizes: an
%       intra-word
%dash, a medium dash for number ranges like
%       1--2,
%and a punctuation
%       dash---like
%this.
%
%A sentence-ending space should be larger than the space between words
%within a sentence.  You sometimes have to type special commands in
%conjunction with punctuation characters to get this right, as in the
%following sentence.
%       Gnats, gnus, etc.\    % `\ ' makes an inter-word space.
%       all begin with G\@.   % \@ marks end-of-sentence punctuation.
%You should check the spaces after periods when reading your output to
%make sure you haven't forgotten any special cases.
%Generating an ellipsis
%       \ldots\    % `\ ' needed because TeX ignores spaces after
%          % command names like \ldots made from \ + letters.
%          %
%          % Note how a `%' character causes TeX to ignore the
%          % end of the input line, so these blank lines do not
%          % start a new paragraph.
%with the right spacing around the periods
%requires a special  command.
%
%\TeX\ interprets some common characters as commands, so you must type
%special commands to generate them.  These characters include the
%following:
%%       \$
%       \& \% \# \{ and~\}.
%
%In printing, text is emphasized by using an
%       {\em italic\/}  % The \/ command produces the tiny extra space that
%               % should be added between a slanted and a following
%               % unslanted letter.
%type style.
%
%\begin{em}
%   A long segment of text can also be emphasized in this way.  Text within
%   such a segment given additional emphasis
%      with\/ {\em Roman}
%   type.  Italic type loses its ability to emphasize and become simply
%   distracting when used excessively.
%\end{em}
%
%It is sometimes necessary to prevent \TeX\ from breaking a line where
%it might otherwise do so.  This may be at a space, as between the
%``Mr.'' and ``Jones'' in
%       ``Mr.~Jones'',        % ~ produces an unbreakable interword space.
%or within a word---especially when the word is a symbol like
%       \mbox{\em itemnum\/}
%that makes little sense when hyphenated across
%       lines.
%
%\TeX\ is good at typesetting mathematical formulas like
%       \( x-3y = 7 \)
%or
%       \( a_{1} > x^{2n} / y^{2n} > x' \).
%Remember that a letter like
%       $x$        % $ ... $  and  \( ... \)  are equivalent
%is a formula when it denotes a mathematical symbol, and should
%be treated as one.
%
%
%\section{Notes}
%Footnotes\footnote{This is an example of a footnote.}
%pose no problem\footnote{And another one}.
%
%\section{Displayed text}
%
%Text is displayed by indenting it from the left margin.
%Quotations are commonly displayed.  There are short quotations
%\begin{quote}
%   This is a short a quotation.  It consists of a
%   single paragraph of text.  There is no paragraph
%   indentation.
%\end{quote}
%and longer ones.
%\begin{quotation}
%   This is a longer quotation.  It consists of two paragraphs
%   of text.  The beginning of each paragraph is indicated
%   by an extra indentation.
%
%   This is the second paragraph of the quotation.  It is just
%   as dull as the first paragraph.
%\end{quotation}
%Another frequently-displayed structure is a list.
%The following is an example of an {\em itemized} list, four levels deep.
%\begin{itemize}
%\item  This is the first item of an itemized list.  Each item
%      in the list is marked with a ``tick''.  The document
%      style determines what kind of tick mark is used.
%\item  This is the second item of the list.  It contains another
%      list nested inside it.  The three inner lists are an {\em itemized}
%      list.
%    \begin{itemize}
%       \item This is the first item of an enumerated list that
%            is nested within the itemized list.
%          \item This is the second item of the inner list.  \LaTeX\
%            allows you to nest lists deeper than you really should.
%      \end{itemize}
%      This is the rest of the second item of the outer list.  It
%      is no more interesting than any other part of the item.
%   \item  This is the third item of the list.
%\end{itemize}
%
%
%The following is an example of an {\em enumerated} list, four levels deep.
%\begin{enumerate}
%\item  This is the first item of an enumerated list.  Each item
%      in the list is marked with a ``tick''.  The document
%      style determines what kind of tick mark is used.
%\item  This is the second item of the list.  It contains another
%      list nested inside it.  The three inner lists are an {\em enumerated}
%      list.
%    \begin{enumerate}
%       \item This is the first item of an enumerated list that
%            is nested within the enumerated list.
%          \item This is the second item of the inner list.  \LaTeX\
%            allows you to nest lists deeper than you really should.
%      \end{enumerate}
%      This is the rest of the second item of the outer list.  It
%      is no more interesting than any other part of the item.
%   \item  This is the third item of the list.
%\end{enumerate}
%
%
%The following is an example of a {\em description} list.
%\begin{description}
%\item[Cow] Highly intelligent animal that can produce milk out of grass.
%\item[Horse] Less intelligent animal renowned for its legs.
%\item[Human being] Not so intelligent animal that thinks that it can think.
%\end{description}
%
%You can even display poetry.
%\begin{verse}
%   There is an environment for verse \\    % The \\ command separates lines
%   Whose features some poets will curse.   % within a stanza.
%
%               % One or more blank lines separate stanzas.
%
%   For instead of making\\
%   Them do {\em all\/} line breaking, \\
%   It allows them to put too many words on a line when they'd
%   rather be forced to be terse.
%\end{verse}
%
%Mathematical formulas may also be displayed.  A displayed formula is
%one-line long; multiline formulas require special formatting
%instructions.
%   \[  x' + y^{2} = z_{i}^{2}\]
%Don't start a paragraph with a displayed equation, nor make
%one a paragraph by itself.
%
%Example of a theorem:
%
%
%\begin{thm}
%All conjectures are interesting, but some conjectures are more
%interesting than others.
%\end{thm}
%
%\begin{proof}
%Obvious.
%\end{proof}
%
%\section{Tables and figures}
%Cross reference to labelled table: As you can see in Table~\ref{sphericcase} on
%page~\pageref{sphericcase} and also in Table~\ref{parset} on page~\pageref{parset}.
%
%
%\begin{table*}
%\caption{The spherical case ($I_1=0$, $I_2=0$).}
%\label{sphericcase}
%\begin{tabular}{crrrrc}
%\hline
%Equil. \\
%Points & \multicolumn{1}{c}{$x$} & \multicolumn{1}{c}{$y$} & \multicolumn{1}{c}{$z$} & \multicolumn{1}{c}{$C$} &
%S \\
%\hline
%$~~L_1$ & $-$2.485252241 & 0.000000000 & 0.017100631 & 8.230711648 & U \\
%$~~L_2$ &    0.000000000 & 0.000000000 & 3.068883732 & 0.000000000 & S \\
%$~~L_3$ &    0.009869059 & 0.000000000 & 4.756386544 & $-$0.000057922 & U \\
%$~~L_4$ &    0.210589855 & 0.000000000 & $-$0.007021459 & 9.440510897 & U \\
%$~~L_5$ &    0.455926604 & 0.000000000 & $-$0.212446624 & 7.586126667 & U \\
%$~~L_6$ &    0.667031314 & 0.000000000 & 0.529879957 & 3.497660052 & U \\
%$~~L_7$ &    2.164386674 & 0.000000000 & $-$0.169308438 & 6.866562449 & U \\
%$~~L_8$ &    0.560414471 & 0.421735658 & $-$0.093667445 & 9.241525367 & U \\
%$~~L_9$ &    0.560414471 & $-$0.421735658 & $-$0.093667445 & 9.241525367 & U
%\\
%$~~L_{10}$ & 1.472523232 & 1.393484549 & $-$0.083801333 & 6.733436505 & U \\
%$~~L_{11}$ & 1.472523232 & $-$1.393484549 & $-$0.083801333 & 6.733436505 & U
%\\ \hline
%\end{tabular}
%\end{table*}
%
%
%A major
%point of difference lies in the value of the specific production rate $\pi$ for
%large values of the specific growth rate $\mu$.
%Already in the early publications \cite{r1,r2,r3}
%it appeared that high glucose
%concentrations in the production phase are well correlated with a
%low penicillin yield (the
%`glucose effect'). It has been confirmed recently
%\cite{r1,r2,r3,r4}
%that
%high glucose concentrations inhibit the synthesis of the enzymes of the
%penicillin pathway, but not the actual penicillin biosynthesis.
%In other words, glucose represses (and not inhibits) the penicillin
%biosynthesis.
%
%These findings do not contradict the results of
%\cite{r1}  and of \cite{r4} which were obtained for
%continuous culture fermentations.
%Because for high values of the specific
%growth rate $\mu$ it is most likely (as shall be discussed below) that
%maintenance metabolism occurs, it can be shown that
%in steady state continuous culture conditions, and with $\mu$ described by a Monod kinetics
%\begin{equation}
%    C_{s}  =  K_{M} \frac{\mu/\mu_{x}}{1-\mu/\mu_{x}} \label{cs}
%\end{equation}
%Pirt \& Rhigelato determined $\pi$ for $\mu$ between
%$0.023$ and $0.086$ h$^{-1}$.
%They also reported a value $\mu_{x} \approx 0.095$
%h$^{-1}$, so that for their experiments $\mu/\mu_{x}$ is in the range
%of $0.24$ to $0.9$.
%Substituting $K _M$ in Eq. (\ref{cs}) by
%the value $K_{M}=1$ g/L as used by \cite{r1}, one finds
%with the above equation $0.3 < C_{s} < 9$ g/L. This agrees well with
%the work of  \cite{r4}, who reported that penicillin biosynthesis
%repression only occurs at glucose concentrations from $C_{s}=10$ g/L on.
%The conclusion is that the glucose concentrations in the experiments of
%Pirt \& Rhigelato probably were too low for glucose repression to be
%detected. The experimental data published by Ryu \& Hospodka
%are not detailed sufficiently to permit a similar analysis.
%
%\begin{table}
%\centering
%\caption{Parameter sets used by Bajpai \& Reu\ss\ }\label{parset}
%\begin{tabular}{lrll}
%\hline
%\multicolumn{2}{l}{\it parameter} & {\it Set 1} & {\it Set 2}\\
%\hline
%$\mu_{x}$           & [h$^{-1}$]  & 0.092       & 0.11          \\
%$K_{x}$             & [g/g DM]     & 0.15        & 0.006         \\
%$\mu_{p}$           & [g/g DM h]  & 0.005       & 0.004         \\
%$K_{p}$             & [g/L]        & 0.0002      & 0.0001        \\
%$K_{i}$             & [g/L]        & 0.1         & 0.1           \\
%$Y_{x/s}$           & [g DM/g]     & 0.45        & 0.47          \\
%$Y_{p/s}$           & [g/g]        & 0.9         & 1.2           \\
%$k_{h}$             & [h$^{-1}$]  & 0.04        & 0.01          \\
%$m_{s}$             & [g/g DM h]  & 0.014       & 0.029         \\
%\hline
%\end{tabular}
%\end{table}
%
%Bajpai \& Reu\ss\ decided to disregard the
%differences between time constants for the two regulation mechanisms
%(glucose repression or inhibition) because of the
%relatively very long fermentation times, and therefore proposed a Haldane
%expression for $\pi$.
%
%It is interesting that simulations with the \cite{r4} model for the
%initial conditions given by these authors indicate that, when the
%remaining substrate is fed at a constant rate, a considerable and
%unrealistic amount of penicillin is
%produced when the glucose concentration is still very high \cite{r2,r3,r4}
%Simulations with the Bajpai \& Reu\ss\ model correctly predict almost
%no penicillin production in similar conditions.
%
%\begin{figure} % figuur 1
%\vspace{6pc}
%\caption[]{Pathway of the penicillin G biosynthesis.}
%\label{penG}
%\end{figure}
%
%Sample of cross-reference to figure.
%Figure~\ref{penG} shows that is not easy to get something on paper.
%
%
%
%\section{Headings}
%
%\subsection{Subsection}
%Carr-Goldstein based their model on balancing methods and
%biochemical know\-ledge. The original model (1980) contained an equation for the
%oxygen dynamics which has been omitted in a second paper
%(1981). This simplified model shall be discussed here.
%
%\subsubsection{Subsubsection}
%Carr-Goldstein
%based their model on balancing methods and
%biochemical know\-ledge. The original model (1980) contained an equation for the
%oxygen dynamics which has been omitted in a second paper
%(1981). This simplified model shall be discussed here.
%
%\section{Equations and the like}
%
%Two equations:
%\begin{equation}
%    C_{s}  =  K_{M} \frac{\mu/\mu_{x}}{1-\mu/\mu_{x}} \label{ccs}
%\end{equation}
%
%and
%
%\begin{equation}
%    G = \frac{P_{\rm opt} - P_{\rm ref}}{P_{\rm ref}} \mbox{\ }100 \mbox{\ }(\%)
%\end{equation}
%
%Two equation arrays:
%
%\begin{eqnarray}
%  \frac{dS}{dt} & = & - \sigma X + s_{F} F\\
%  \frac{dX}{dt} & = &   \mu    X\\
%  \frac{dP}{dt} & = &   \pi    X - k_{h} P\\
%  \frac{dV}{dt} & = &   F
%\end{eqnarray}
%
%and,
%
%\begin{eqnarray}
% \mu_{\rm substr} & = & \mu_{x} \frac{C_{s}}{K_{x}C_{x}+C_{s}}  \\
% \mu              & = & \mu_{\rm substr} - Y_{x/s}(1-H(C_{s}))(m_{s}+\pi /Y_{p/s}) \\
% \sigma           & = & \mu_{\rm substr}/Y_{x/s}+ H(C_{s}) (m_{s}+ \pi /Y_{p/s})
%\end{eqnarray}
%
%
%\appendix
%
%\section{Appendix section}\label{app}
%
%We consider a sequence of queueing systems
%indexed by $n$.  It is assumed that each system
%is composed of $J$ stations, indexed by $1$
%through $J$, and $K$ customer classes, indexed
%by $1$ through $K$.  Each customer class
%has a fixed route through the network of
%stations.  Customers in class
%$k$, $k=1,\ldots,K$, arrive to the
%system according to a
%renewal process, independently of the arrivals
%of the other customer classes.  These customers
%move through the network, never visiting a station
%more than once, until they eventually exit
%the system.
%
%\subsection{Appendix subsection}
%
%However, different customer classes may visit
%stations in different orders; the system
%is not necessarily ``feed-forward.''
%We define the {\em path of class $k$ customers} in
%as the sequence of servers
%they encounter along their way through the network
%and denote it by
%\begin{equation}
%\mathcal{P}=\bigl(j_{k,1},j_{k,2},\dots,j_{k,m(k)}\bigr). \label{path}
%\end{equation}
%
%Sample of cross-reference to the formula \ref{path} in Appendix \ref{app}.

%\section*{Acknowledgements}
%And this is an acknowledgements section with a heading that was produced by the
%$\backslash$section* command. Thank you all for helping me writing this
%\LaTeX\ sample file.

%\begin{thebibliography}{9}
%
%\bibitem{r1}
%\textsc{Billingsley, P.} (1999). \textit{Convergence of
%Probability Measures}, 2nd ed.
%Wiley, New York.
%\MR{1700749}
%
%
%\bibitem{r2}
%\textsc{Bourbaki, N.}  (1966). \textit{General Topology}  \textbf{1}.
%Addison--Wesley, Reading, MA.
%
%\bibitem{r3}
%\textsc{Ethier, S. N.} and \textsc{Kurtz, T. G.} (1985).
%\textit{Markov Processes: Characterization and Convergence}.
%Wiley, New York.
%\MR{838085}
%
%\bibitem{r4}
%\textsc{Prokhorov, Yu.} (1956).
%Convergence of random processes and limit theorems in probability
%theory. \textit{Theory  Probab.  Appl.}
%\textbf{1} 157--214.
%\MR{84896}
%
%\end{thebibliography}

\end{document}
