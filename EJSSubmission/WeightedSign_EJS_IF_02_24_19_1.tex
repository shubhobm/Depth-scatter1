\section{Influence Functions of Dispersion Measures}
\label{Sec:RE_Dispersion}


We retain the framework adopted in Section~\ref{Sec:WSDispersion1}, and discuss in this 
section the robustness and efficiency properties associated with $\tilde{\Sigma}$ 
and $\Sigma_{*}$, and principal components derived therefrom. 
We do not discuss $\Sigma^{\dagger}$ here, since  the 
properties of that approach follow from those of  $\tilde{\Sigma}$. 
We additionally assume that the eigenvalues of $\Sigma$ are 
distinct, and given by  $\lambda_1 > \lambda_2 > \ldots > \lambda_p$, to avoid 
several additional technical conditions for the theoretical results to follow. 
The case where the eigenvalues of $\Sigma$ can have multiplicity greater than one 
requires no additional conceptual development, but does require considerable algebraic 
manipulations.

For studying 
the robustness aspect, we first present some results relating to influence functions 
in the current context. 
 Influence functions quantify how much influence a sample point, especially 
an infinitesimal contamination, has on any functional of a probability 
distribution \citep{ref:HampelBook86}. Given any probability distribution 
$\BH \in \cM$, the influence function of any point $x_0 \in \mathcal{X}$ for 
some functional $T(\BH)$ on the distribution is defined as
%
\ban
IF(x_0; T,\BH) =
\lim_{\epsilon \rightarrow 0} \frac{1}{\epsilon} (T(\BH_\epsilon) - T(\BH)),
\ean
%
where 
$\BH_\epsilon = (1-\epsilon)\BH + \epsilon \delta_{x_0}$; $\delta_{x_0}$ being 
the distribution with point mass at $x_0$. When $T(\BH) = E_\BH f$ for some 
$\BH$-integrable function $f$, $IF(x_0; T,\BH) = f(x_0) - T(\BH)$.

It now follows that
%
\ban
IF (x_0; \tilde{\Sigma}, \BF) =
W^{2} (x_{0}) \BS(x_0; \mu) - \tilde \Sigma.
\ean
%
Recall that $\tilde{\lambda}_{1} > \tilde{\lambda}_{2} > \ldots > \tilde{\lambda}_{p}$
are the eigenvalues of $\tilde{\Sigma}$, which we assume are all distinct values. 

\begin{Proposition}\label{Thm:IF}
The influence function of $\bfgamma_{i}$ as follows: 
\ban
IF(x_0; \bfgamma_{i}, \BF)  & = 
\sum_{k = 1; k \neq i}^p \frac{1}{\tilde{\lambda}_{i} - \tilde{\lambda}_{k}} 
\left\{ \bfgamma^T_k IF(x_0; \tilde \Sigma,\BF)  \bfgamma_{i} \right\} 
\bfgamma_k \notag \\
& =  \sum_{k=1; k \neq i}^p \frac{1}{\tilde{\lambda}_{i} - \tilde{\lambda}_{k}}
W^{2} (x_{0}) \left\{  \bfgamma^T_{k} \BS (x_0; \mu) \bfgamma_{i} \right\} 
\bfgamma_{k}.
\ean
\end{Proposition}

The proof of Proposition~\ref{Thm:IF} follows from \cite{ref:JRSSB79217_Sibson} and 
\cite{ref:Biometrika00603_CrouxHaesbroeck}, we omit the details. 

If the weight function $W (\cdot)$ is a bounded function, as is the case of $W_{HSD}$, 
$W_{MhD}$, and $W_{PD}$, 
the influence function given in  Proposition~\ref{Thm:IF} is bounded, indicating good 
robustness properties of the principal component analysis.



We now derive the influence function for $\Sigma_{*}$. 
\begin{Proposition}\label{Thm:IF2}
The influence function of $\Sigma_{*}$ is given by
\ban 
IF ( x_{0}, \Sigma_{*}, \BF) = 
\alpha_{\Sigma_{*}} ( | x_{0} |; \BF_{Z} ) 
\BS( x_0; \mu) 
- \beta_{\Sigma_{*}} (| x_{0} |; \BF_{Z} ) \Sigma_{*}.
\ean
\end{Proposition}

\begin{proof}[Proof of Proposition~\ref{Thm:IF2}:]

Let $z_{0} = \Lambda^{-1/2} \Gamma^T  (x_0 - \mu) = (z_{0 1}, \ldots, z_{0 p})^T$.
As a first step, since $\Sigma_{*}$ is affine equivariant, we obtain from 
\cite{ref:Biometrika00603_CrouxHaesbroeck} 
that 
\ban 
IF ( x_{0}, \Sigma_{*}, \BF) = \Sigma_{*}^{1/2} 
IF ( z_{0}, \Sigma_{*}, \BF_{Z})  \Sigma_{*}^{1/2}.
\ean
From Lemma1 of \citep{ref:HampelBook86}, page 276, we obtain that there exist 
scalar valued functions $\alpha_{\Sigma_{*}} ( | x_{0} |; \BF_{Z} )$ and 
$\beta_{\Sigma_{*}} ( | x_{0} |; \BF_{Z} ) $ such  that 
\ban
IF ( z_{0}, \Sigma_{*}, \BF_{Z}) = \alpha_{\Sigma_{*}} ( | x_{0} |; \BF_{Z} ) 
\BS( z_0; {\mathbf 0}) 
- \beta_{\Sigma_{*}} (| x_{0} |; \BF_{Z} ) \BI_{p}, 
\ean
consequently we obtain
\ban 
IF ( x_{0}, \Sigma_{*}, \BF) = 
\alpha_{\Sigma_{*}} ( | x_{0} |; \BF_{Z} ) 
\BS( x_0; \mu) 
- \beta_{\Sigma_{*}} (| x_{0} |; \BF_{Z} ) \Sigma_{*}.
\ean
\end{proof}

Suppose ${\lambda}_{* 1} > {\lambda}_{* 2} > \ldots > {\lambda}_{* p}$
are the eigenvalues of ${\Sigma}_{*}$, which we assume are all distinct values. 
Also denote the $i$-th eigenvector of $\Sigma_{*}$ by 
$\bfgamma_{* i} = (\gamma_{* i 1}, \ldots, \gamma_{* i p})^T$ for $1 \leq i \leq p$.

\begin{Proposition}\label{Thm:IF3}
The influence function of $\bfgamma_{* i} $ may be obtained as 
\ban
IF(x_0; \bfgamma_{*i}, \BF)  & = 
\sum_{k = 1; k \neq i}^p \frac{1}{{\lambda}_{*i} - {\lambda}_{*k}} 
\left\{ \bfgamma^T_{* k} IF(x_0; \tilde \Sigma,\BF)  \bfgamma_{* i} \right\} 
\bfgamma_{* k} \notag \\
& =  
\alpha_{\Sigma_{*}} ( | x_{0} |; \BF_{Z} ) \sum_{k=1; k \neq i}^p 
\frac{1}{{\lambda}_{*i} - {\lambda}_{*k}}
\left\{ \bfgamma^T_{* k} \BS (x_0; \mu) \bfgamma_{* i} \right\} 
\bfgamma_{* k}.
\ean
\end{Proposition}

We omit the proof of Proposition~\ref{Thm:IF3}, which follows along similar lines to 
to rest of the computations of this section.
It can be shown that when $W (\cdot)$ is a bounded function, 
$\alpha_{\Sigma_{*}} ( | x_{0} |; \BF_{Z} )$ is also bounded, along the lines of 
\cite{ref:HuberBook81}, which in turn implies that the influence function for a 
principal component based on $\Sigma_{*}$ is also bounded.


