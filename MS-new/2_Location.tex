\section{Weighted sign vectors in location problem}
\label{section:LocSection}
In this section, we focus on the general situation of estimating or testing for the location parameter $\bfmu \equiv \bfmu_F$ using the weighted sign vectors in \eqref{eqn:rank}. We impose the following very general conditions on the weights $w(\bfx, \BF) := \kappa_P(P(\bfx, \BF))$:

\vspace{1em}
\noindent{\bf (A1)} The weights are affine invariant, i.e.
%
$$
w(\bfA \bfx + \bfb, \bfA \BF + \bfb) = w(\bfx, \BF)
$$
%
for any $\bfb \in \BR^p$ and full rank matrix $\bfA \in \BR^{p \times p}$.

\noindent{\bf (A2)} Given a fixed $\BF$, the weights are square-integrable:
%
$$
\int [w(\bfy, \BF)]^2 d\BG(\bfy) < \infty; \BG \in \cM.
$$

\vspace{1em}
\noindent Because of the affine invariance, it is possible to simplify the weight as a function of the norm of $\bfz = \Sigma^{-1/2} (\bfX - \bfmu)$: $w(\bfx, \BF) = f(r)$, with $r = \| \bfz \|$.

A straightforward use of weighted signs in a location problem is to construct an outlier-robust alternative to the Hotelling's $T^2$ test using their sample mean vector and covariance matrix. Formally, this means testing for $H_0: \bfmu = {\bf 0}_p$ vs. $H_1:\bfmu \neq {\bf 0}_p$  based on the test statistic:
%
$$
T_{n,w} = n \bar\bfx_w^T ( \BS (\BX_w))^{-1} \bar\bfx_w,
$$
%
with $\bar\bfx_w = \sum_{i=1}^n \bfx_{w,i}/n, \bfx_{w,i} = w(\bfx_i ) \bfS (\bfx_i)$ for $i=1,2,...,n$, and $\BX_w = (\bfx_{w,1}, \ldots, \bfx_{w,n})^T$. However, the following holds true for this weighted sign test:
%
\begin{Proposition}\label{proposition:SignTest}
Consider $n$ random variables $\bfZ_1,...,\bfZ_n$ distributed independently and identically as $\mathcal{E}( \bfmu, k \bfI_p, g); k \in \mathbb R$, and the class of hypothesis tests defined above. Then, given any $\alpha \in (0,1)$, local power at $\bfmu \neq {\bf 0}_p$ for the level-$\alpha$ test  based on $T_{n,w}$ is maximum when $w(\bfZ_1) = c$, a constant independent of $\bfZ_1$.
\end{Proposition}
%
\noindent This essentially means that power-wise the (unweighted) spatial sign test \citep{OjaBook10} is optimal in the given class of hypothesis tests when the data comes from a spherically symmetric distribution. Our simulations show that this empirically holds for non-spherical but elliptic distributions as well.

\subsection{The weighted spatial median} 
To explore usage of weighted spatial signs in the location problem that improve upon the state-of-the-art, we concentrate on the following optimization problem:
%
\begin{equation}\label{eqn:WtSpMed}
\bfmu_w = \argmin_{\bfmu_0 \in \mathbb{R}^p} \BE ( w(\bfX) | \bfX - \bfmu_0 |).
\end{equation}
%
This can be seen as a generalization of the Fermat-Weber location problem (which has the spatial median \citep{brown83, Chaudhuri96} as the solution) using data-dependent weights. Using affine equivariant weights in \eqref{eqn:WtSpMed} ensures that the weights are independent of $\bfmu_0$, allowing the optimization problem to have a unique solution. We call this solution the \textit{weighted spatial median} of $\BF$, and denote it by $\bfmu_w$. In a sample setup it is estimated by iteratively solving the equation $\sum_{i=1}^n w(\bfX_i) \bfS (\bfX_i - \hat\bfmu_w)/n = {\bf 0}_p$.

The sample weighted spatial median $\hat\bfmu_w$ is a $\sqrt n$-consistent estimator of $\bfmu_w$, with the following result gives its asymptotic distribution:
%
\begin{Theorem}
Let $\BA_w, \BB_w$ be two matrices, dependent on the weight function $w$ such that
%
$$
\BA_w = \BE \left[ \frac{w( \bfepsilon ) }{\| \bfepsilon \|} \left( 1 - \frac{\bfepsilon \bfepsilon^T}{\| \bfepsilon \|^2} \right) \right];
\quad
\BB_w = \BE \left[ \frac{(w( \bfepsilon ))^2 \bfepsilon \bfepsilon^T}{\| \bfepsilon \|^2} \right],
$$
%
where $\bfepsilon \sim \mathcal E({\bf 0}, \Sigma, g)$. Then
%
\begin{equation}
\sqrt n (\hat\bfmu_w - \bfmu_w) \leadsto \cN_p ({\bf 0}, \BA_w^{-1} \BB_w \BA_w^{-1}).
\end{equation}
\end{Theorem}
%

%We provide a sketch of its proof in the supplementary material, which generalizes equivalent results for the spatial median \citep{OjaBook10}. 
Setting $w(\bfepsilon)=1$ above yields the asymptotic covariance matrix for the spatial median. Following this, the asymptotic relative efficiency (ARE) of $\bfmu_w$ corresponding to some non-uniform weight function with respect to the spatial median, say $\bfmu_s$ will be:
%
$$
ARE( \bfmu_w, \bfmu_s) = \left[ \frac{\text{det} ( \BA^{-1} \BB \BA^{-1})}{\text{det} ( \BA_w^{-1} \BB_w \BA_w^{-1})} \right]^{1/p}
$$
%
with $\BA = \BE [ 1/ \| \bfepsilon \| ( \BI_p - \bfepsilon \bfepsilon^T/ \| \bfepsilon \|^2 ) ]$ and $\BB = \BE [ \bfepsilon \bfepsilon^T/ \| \bfepsilon \|^2 ]$. This is further simplified under spherical symmetry:

\begin{Corollary}
For the spherical distribution $\mathcal{E}(\bfmu, k \bfI_p, g); k \in \mathbb R, \bfmu \in \mathbb R^p$, we have
%
$$
ARE( \bfmu_w, \bfmu_s) = \frac{ \left[ \BE \left( \frac{f(r)}{r} \right) \right]^2}{\BE f^2(r) \left[ \BE \left( \frac{1}{r} \right) \right]^2 }.
$$
\end{Corollary}
%
\begin{table}[t]
\centering
\begin{footnotesize}
\begin{tabular}{c|ccccc}
    \hline
    & $t_3$   & $t_5$   & $t_{10}$  & $t_{20}$  & Normal \\ \hline
    $p=5$    & 1.28 & 1.20 & 1.16 & 1.14 & 1.13   \\
    $p=10$   & 1.15 & 1.10 & 1.07 & 1.07 & 1.06   \\
    $p=20$   & 1.09 & 1.05 & 1.04 & 1.03 & 1.03   \\
    $p=50$   & 1.05 & 1.02 & 1.01 & 1.01 & 1.01   \\ \hline
\end{tabular}
\end{footnotesize}
\caption{Table of $ARE(\bfmu_w; \bfmu_s)$ for different spherical distributions}
\label{table:AREtablewsm}
\end{table}
%
Table \ref{table:AREtablewsm} summarizes the AREs for several families of elliptic distributions, numerically calculated using 10,000 random samples, and taking $f(r) = 1/(1+r)$. It is evident from the table that the weighted spatial median outperforms its unweighted counterpart for all data dimensions and distribution families. While the performance is much better for small values of $p$, weighting the signs seems to have less and less effect as $p$ grows larger. Assuming a first-order autoregressive (AR(1)) covariance structure, i.e. $\sigma_{ij} = \rho^{|i-j|}$ with $ \rho \in (0,1)$, results in largely similar ARE values as those obtained in table \ref{table:AREtablewsm} that assume $\Sigma = \bfI_p$.

\subsection{A high-dimensional test of location}

It is possible to take an alternative approach to the location testing problem by using the covariance-type U-statistic $C_{n,w} = \sum_{i=1}^n \sum_{j=1}^{i-1} \bfX_{w,i}^T \bfX_{w,j}$. This class of test statistics are especially attractive since they are readily generalized to cover high-dimensional situations, i.e. when $p > n$. The Chen and Qin (CQ) high-dimensional test of location for multivariate normal $\bfX_i$ \citep{ChenQin10} is a special case of this test that uses the statistic $C_n = \sum_{i=1}^n \sum_{j=1}^{i-1} \bfX_i^T \bfX_j$. The testing procedure of \cite{WangPengLi15} (from here on referred to as WPL test) shows that one can improve upon the power of the CQ test for non-gaussian elliptical distributions by using spatial signs $\bfS(\bfX_i)$ in place of the actual variables.

Under mild regularity conditions in the lines of \cite{WangPengLi15}, the following results hold for our generalized test statistic $C_{n,w}$ under $H_0$ as $n,p \rightarrow \infty$:
%
\begin{equation}\label{eqn:hdtest1}
\frac{C_{n,w}}{\sqrt{\frac{n(n-1)}{2} \text{Tr}(\BB_w^2)}} \leadsto N(0,1),
\end{equation}
%
and under contiguous alternatives $H_1: \bfmu = \bfmu_0$,
%
\begin{equation}\label{eqn:hdtest2}
\frac{C_{n,w} - \frac{n(n-1)}{2} \bfmu_0^T \BA_w^2 \bfmu_0 (1 + o(1)) }{\sqrt{\frac{n(n-1)}{2} \text{Tr}(\BB_w^2)}} \leadsto N(0,1).
\end{equation}
%
We provide the details behind deriving these two results in the supplementary material, which extend the results of \cite{WangPengLi15} in a weightd sign setup using modified regularity conditions.

Following this, the ARE of this test statistic with respect to its unweighted version, i.e. the WPL statistic, is expressed as:
%
$$
ARE(C_{n,w}, \text{WPL}; \bfmu_0) = \frac{\bfmu_0^T \BA_w^2 \bfmu_0}{\bfmu_0^T \BA^2 \bfmu_0} \sqrt\frac{\text{Tr}( \BB^2)}{\text{Tr}( \BB_w^2)} (1 + o(1)),
$$
%
when $\Sigma = k \BI_p$, this again simplifies to $\BE^2(f(r)/r)/[\BE f^2(r). \BE^2(1/r)]$. Thus the ARE values are exactly same as those in Table \ref{table:AREtablewsm}, indicating that for large data dimensions the WPL test and that based on $C_{n,w}$ are almost equivalent.

However, in a practical high-dimensional setup sample sizes are often small. Thus, comparing the the two tests with respect to their \textit{finite sample} efficiencies instead gives a better idea of their practical utility. We do so in table~\ref{table:AREtablehd}, which lists empirical powers calculated from 1000 replications of each setup under an AR(1) covariance structure (with $\rho = 0.8$). While under $H_0: \bfmu = {\bf 0}_p$ all tests have similar performance, $C_{n,w}$ beats the other two under deviations from the null.

\begin{table}
\centering
\begin{footnotesize}
\begin{tabular}{cc|ccc}\hline
\multicolumn{5}{l}{$\bfmu = \text{rep}(.15,p)$}\\\hline 
 $p$  & $n$    & CQ   & WPL  & $C_{n,w}$ \\\hline 
  500 & 20 & 0.051 & 0.376 & 0.418 \\
  500 & 50 & 0.060 & 0.832 & 0.866 \\
 1000 & 20 & 0.044 & 0.541 & 0.584 \\
 1000 & 50 & 0.039 & 0.973 & 0.987 \\\hline
\multicolumn{5}{l}{$\bfmu = \text{rep}(0,p)$}\\\hline
 $p$  & $n$    & CQ   & WPL  & $C_{n,w}$ \\\hline 
  500 & 20 & 0.049 & 0.061 & 0.063 \\
  500 & 50 & 0.039 & 0.061 & 0.064 \\
 1000 & 20 & 0.042 & 0.060 & 0.063 \\
 1000 & 50 & 0.043 & 0.050 & 0.050 \\\hline
\end{tabular}
\end{footnotesize}
\caption{Table of empirical powers of level-0.05 tests for the Chen and Qin (CQ), WPL and $C_{n,w}$ statistics}
\label{table:AREtablehd}
\end{table}
